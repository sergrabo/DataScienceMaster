
% Default to the notebook output style

    


% Inherit from the specified cell style.




    
\documentclass[11pt]{article}

    
    
    \usepackage[T1]{fontenc}
    % Nicer default font (+ math font) than Computer Modern for most use cases
    \usepackage{mathpazo}

    % Basic figure setup, for now with no caption control since it's done
    % automatically by Pandoc (which extracts ![](path) syntax from Markdown).
    \usepackage{graphicx}
    % We will generate all images so they have a width \maxwidth. This means
    % that they will get their normal width if they fit onto the page, but
    % are scaled down if they would overflow the margins.
    \makeatletter
    \def\maxwidth{\ifdim\Gin@nat@width>\linewidth\linewidth
    \else\Gin@nat@width\fi}
    \makeatother
    \let\Oldincludegraphics\includegraphics
    % Set max figure width to be 80% of text width, for now hardcoded.
    \renewcommand{\includegraphics}[1]{\Oldincludegraphics[width=.8\maxwidth]{#1}}
    % Ensure that by default, figures have no caption (until we provide a
    % proper Figure object with a Caption API and a way to capture that
    % in the conversion process - todo).
    \usepackage{caption}
    \DeclareCaptionLabelFormat{nolabel}{}
    \captionsetup{labelformat=nolabel}

    \usepackage{adjustbox} % Used to constrain images to a maximum size 
    \usepackage{xcolor} % Allow colors to be defined
    \usepackage{enumerate} % Needed for markdown enumerations to work
    \usepackage{geometry} % Used to adjust the document margins
    \usepackage{amsmath} % Equations
    \usepackage{amssymb} % Equations
    \usepackage{textcomp} % defines textquotesingle
    % Hack from http://tex.stackexchange.com/a/47451/13684:
    \AtBeginDocument{%
        \def\PYZsq{\textquotesingle}% Upright quotes in Pygmentized code
    }
    \usepackage{upquote} % Upright quotes for verbatim code
    \usepackage{eurosym} % defines \euro
    \usepackage[mathletters]{ucs} % Extended unicode (utf-8) support
    \usepackage[utf8x]{inputenc} % Allow utf-8 characters in the tex document
    \usepackage{fancyvrb} % verbatim replacement that allows latex
    \usepackage{grffile} % extends the file name processing of package graphics 
                         % to support a larger range 
    % The hyperref package gives us a pdf with properly built
    % internal navigation ('pdf bookmarks' for the table of contents,
    % internal cross-reference links, web links for URLs, etc.)
    \usepackage{hyperref}
    \usepackage{longtable} % longtable support required by pandoc >1.10
    \usepackage{booktabs}  % table support for pandoc > 1.12.2
    \usepackage[inline]{enumitem} % IRkernel/repr support (it uses the enumerate* environment)
    \usepackage[normalem]{ulem} % ulem is needed to support strikethroughs (\sout)
                                % normalem makes italics be italics, not underlines
    

    
    
    % Colors for the hyperref package
    \definecolor{urlcolor}{rgb}{0,.145,.698}
    \definecolor{linkcolor}{rgb}{.71,0.21,0.01}
    \definecolor{citecolor}{rgb}{.12,.54,.11}

    % ANSI colors
    \definecolor{ansi-black}{HTML}{3E424D}
    \definecolor{ansi-black-intense}{HTML}{282C36}
    \definecolor{ansi-red}{HTML}{E75C58}
    \definecolor{ansi-red-intense}{HTML}{B22B31}
    \definecolor{ansi-green}{HTML}{00A250}
    \definecolor{ansi-green-intense}{HTML}{007427}
    \definecolor{ansi-yellow}{HTML}{DDB62B}
    \definecolor{ansi-yellow-intense}{HTML}{B27D12}
    \definecolor{ansi-blue}{HTML}{208FFB}
    \definecolor{ansi-blue-intense}{HTML}{0065CA}
    \definecolor{ansi-magenta}{HTML}{D160C4}
    \definecolor{ansi-magenta-intense}{HTML}{A03196}
    \definecolor{ansi-cyan}{HTML}{60C6C8}
    \definecolor{ansi-cyan-intense}{HTML}{258F8F}
    \definecolor{ansi-white}{HTML}{C5C1B4}
    \definecolor{ansi-white-intense}{HTML}{A1A6B2}

    % commands and environments needed by pandoc snippets
    % extracted from the output of `pandoc -s`
    \providecommand{\tightlist}{%
      \setlength{\itemsep}{0pt}\setlength{\parskip}{0pt}}
    \DefineVerbatimEnvironment{Highlighting}{Verbatim}{commandchars=\\\{\}}
    % Add ',fontsize=\small' for more characters per line
    \newenvironment{Shaded}{}{}
    \newcommand{\KeywordTok}[1]{\textcolor[rgb]{0.00,0.44,0.13}{\textbf{{#1}}}}
    \newcommand{\DataTypeTok}[1]{\textcolor[rgb]{0.56,0.13,0.00}{{#1}}}
    \newcommand{\DecValTok}[1]{\textcolor[rgb]{0.25,0.63,0.44}{{#1}}}
    \newcommand{\BaseNTok}[1]{\textcolor[rgb]{0.25,0.63,0.44}{{#1}}}
    \newcommand{\FloatTok}[1]{\textcolor[rgb]{0.25,0.63,0.44}{{#1}}}
    \newcommand{\CharTok}[1]{\textcolor[rgb]{0.25,0.44,0.63}{{#1}}}
    \newcommand{\StringTok}[1]{\textcolor[rgb]{0.25,0.44,0.63}{{#1}}}
    \newcommand{\CommentTok}[1]{\textcolor[rgb]{0.38,0.63,0.69}{\textit{{#1}}}}
    \newcommand{\OtherTok}[1]{\textcolor[rgb]{0.00,0.44,0.13}{{#1}}}
    \newcommand{\AlertTok}[1]{\textcolor[rgb]{1.00,0.00,0.00}{\textbf{{#1}}}}
    \newcommand{\FunctionTok}[1]{\textcolor[rgb]{0.02,0.16,0.49}{{#1}}}
    \newcommand{\RegionMarkerTok}[1]{{#1}}
    \newcommand{\ErrorTok}[1]{\textcolor[rgb]{1.00,0.00,0.00}{\textbf{{#1}}}}
    \newcommand{\NormalTok}[1]{{#1}}
    
    % Additional commands for more recent versions of Pandoc
    \newcommand{\ConstantTok}[1]{\textcolor[rgb]{0.53,0.00,0.00}{{#1}}}
    \newcommand{\SpecialCharTok}[1]{\textcolor[rgb]{0.25,0.44,0.63}{{#1}}}
    \newcommand{\VerbatimStringTok}[1]{\textcolor[rgb]{0.25,0.44,0.63}{{#1}}}
    \newcommand{\SpecialStringTok}[1]{\textcolor[rgb]{0.73,0.40,0.53}{{#1}}}
    \newcommand{\ImportTok}[1]{{#1}}
    \newcommand{\DocumentationTok}[1]{\textcolor[rgb]{0.73,0.13,0.13}{\textit{{#1}}}}
    \newcommand{\AnnotationTok}[1]{\textcolor[rgb]{0.38,0.63,0.69}{\textbf{\textit{{#1}}}}}
    \newcommand{\CommentVarTok}[1]{\textcolor[rgb]{0.38,0.63,0.69}{\textbf{\textit{{#1}}}}}
    \newcommand{\VariableTok}[1]{\textcolor[rgb]{0.10,0.09,0.49}{{#1}}}
    \newcommand{\ControlFlowTok}[1]{\textcolor[rgb]{0.00,0.44,0.13}{\textbf{{#1}}}}
    \newcommand{\OperatorTok}[1]{\textcolor[rgb]{0.40,0.40,0.40}{{#1}}}
    \newcommand{\BuiltInTok}[1]{{#1}}
    \newcommand{\ExtensionTok}[1]{{#1}}
    \newcommand{\PreprocessorTok}[1]{\textcolor[rgb]{0.74,0.48,0.00}{{#1}}}
    \newcommand{\AttributeTok}[1]{\textcolor[rgb]{0.49,0.56,0.16}{{#1}}}
    \newcommand{\InformationTok}[1]{\textcolor[rgb]{0.38,0.63,0.69}{\textbf{\textit{{#1}}}}}
    \newcommand{\WarningTok}[1]{\textcolor[rgb]{0.38,0.63,0.69}{\textbf{\textit{{#1}}}}}
    
    
    % Define a nice break command that doesn't care if a line doesn't already
    % exist.
    \def\br{\hspace*{\fill} \\* }
    % Math Jax compatability definitions
    \def\gt{>}
    \def\lt{<}
    % Document parameters
    \title{01-metadataIntro}
    
    
    

    % Pygments definitions
    
\makeatletter
\def\PY@reset{\let\PY@it=\relax \let\PY@bf=\relax%
    \let\PY@ul=\relax \let\PY@tc=\relax%
    \let\PY@bc=\relax \let\PY@ff=\relax}
\def\PY@tok#1{\csname PY@tok@#1\endcsname}
\def\PY@toks#1+{\ifx\relax#1\empty\else%
    \PY@tok{#1}\expandafter\PY@toks\fi}
\def\PY@do#1{\PY@bc{\PY@tc{\PY@ul{%
    \PY@it{\PY@bf{\PY@ff{#1}}}}}}}
\def\PY#1#2{\PY@reset\PY@toks#1+\relax+\PY@do{#2}}

\expandafter\def\csname PY@tok@w\endcsname{\def\PY@tc##1{\textcolor[rgb]{0.73,0.73,0.73}{##1}}}
\expandafter\def\csname PY@tok@c\endcsname{\let\PY@it=\textit\def\PY@tc##1{\textcolor[rgb]{0.25,0.50,0.50}{##1}}}
\expandafter\def\csname PY@tok@cp\endcsname{\def\PY@tc##1{\textcolor[rgb]{0.74,0.48,0.00}{##1}}}
\expandafter\def\csname PY@tok@k\endcsname{\let\PY@bf=\textbf\def\PY@tc##1{\textcolor[rgb]{0.00,0.50,0.00}{##1}}}
\expandafter\def\csname PY@tok@kp\endcsname{\def\PY@tc##1{\textcolor[rgb]{0.00,0.50,0.00}{##1}}}
\expandafter\def\csname PY@tok@kt\endcsname{\def\PY@tc##1{\textcolor[rgb]{0.69,0.00,0.25}{##1}}}
\expandafter\def\csname PY@tok@o\endcsname{\def\PY@tc##1{\textcolor[rgb]{0.40,0.40,0.40}{##1}}}
\expandafter\def\csname PY@tok@ow\endcsname{\let\PY@bf=\textbf\def\PY@tc##1{\textcolor[rgb]{0.67,0.13,1.00}{##1}}}
\expandafter\def\csname PY@tok@nb\endcsname{\def\PY@tc##1{\textcolor[rgb]{0.00,0.50,0.00}{##1}}}
\expandafter\def\csname PY@tok@nf\endcsname{\def\PY@tc##1{\textcolor[rgb]{0.00,0.00,1.00}{##1}}}
\expandafter\def\csname PY@tok@nc\endcsname{\let\PY@bf=\textbf\def\PY@tc##1{\textcolor[rgb]{0.00,0.00,1.00}{##1}}}
\expandafter\def\csname PY@tok@nn\endcsname{\let\PY@bf=\textbf\def\PY@tc##1{\textcolor[rgb]{0.00,0.00,1.00}{##1}}}
\expandafter\def\csname PY@tok@ne\endcsname{\let\PY@bf=\textbf\def\PY@tc##1{\textcolor[rgb]{0.82,0.25,0.23}{##1}}}
\expandafter\def\csname PY@tok@nv\endcsname{\def\PY@tc##1{\textcolor[rgb]{0.10,0.09,0.49}{##1}}}
\expandafter\def\csname PY@tok@no\endcsname{\def\PY@tc##1{\textcolor[rgb]{0.53,0.00,0.00}{##1}}}
\expandafter\def\csname PY@tok@nl\endcsname{\def\PY@tc##1{\textcolor[rgb]{0.63,0.63,0.00}{##1}}}
\expandafter\def\csname PY@tok@ni\endcsname{\let\PY@bf=\textbf\def\PY@tc##1{\textcolor[rgb]{0.60,0.60,0.60}{##1}}}
\expandafter\def\csname PY@tok@na\endcsname{\def\PY@tc##1{\textcolor[rgb]{0.49,0.56,0.16}{##1}}}
\expandafter\def\csname PY@tok@nt\endcsname{\let\PY@bf=\textbf\def\PY@tc##1{\textcolor[rgb]{0.00,0.50,0.00}{##1}}}
\expandafter\def\csname PY@tok@nd\endcsname{\def\PY@tc##1{\textcolor[rgb]{0.67,0.13,1.00}{##1}}}
\expandafter\def\csname PY@tok@s\endcsname{\def\PY@tc##1{\textcolor[rgb]{0.73,0.13,0.13}{##1}}}
\expandafter\def\csname PY@tok@sd\endcsname{\let\PY@it=\textit\def\PY@tc##1{\textcolor[rgb]{0.73,0.13,0.13}{##1}}}
\expandafter\def\csname PY@tok@si\endcsname{\let\PY@bf=\textbf\def\PY@tc##1{\textcolor[rgb]{0.73,0.40,0.53}{##1}}}
\expandafter\def\csname PY@tok@se\endcsname{\let\PY@bf=\textbf\def\PY@tc##1{\textcolor[rgb]{0.73,0.40,0.13}{##1}}}
\expandafter\def\csname PY@tok@sr\endcsname{\def\PY@tc##1{\textcolor[rgb]{0.73,0.40,0.53}{##1}}}
\expandafter\def\csname PY@tok@ss\endcsname{\def\PY@tc##1{\textcolor[rgb]{0.10,0.09,0.49}{##1}}}
\expandafter\def\csname PY@tok@sx\endcsname{\def\PY@tc##1{\textcolor[rgb]{0.00,0.50,0.00}{##1}}}
\expandafter\def\csname PY@tok@m\endcsname{\def\PY@tc##1{\textcolor[rgb]{0.40,0.40,0.40}{##1}}}
\expandafter\def\csname PY@tok@gh\endcsname{\let\PY@bf=\textbf\def\PY@tc##1{\textcolor[rgb]{0.00,0.00,0.50}{##1}}}
\expandafter\def\csname PY@tok@gu\endcsname{\let\PY@bf=\textbf\def\PY@tc##1{\textcolor[rgb]{0.50,0.00,0.50}{##1}}}
\expandafter\def\csname PY@tok@gd\endcsname{\def\PY@tc##1{\textcolor[rgb]{0.63,0.00,0.00}{##1}}}
\expandafter\def\csname PY@tok@gi\endcsname{\def\PY@tc##1{\textcolor[rgb]{0.00,0.63,0.00}{##1}}}
\expandafter\def\csname PY@tok@gr\endcsname{\def\PY@tc##1{\textcolor[rgb]{1.00,0.00,0.00}{##1}}}
\expandafter\def\csname PY@tok@ge\endcsname{\let\PY@it=\textit}
\expandafter\def\csname PY@tok@gs\endcsname{\let\PY@bf=\textbf}
\expandafter\def\csname PY@tok@gp\endcsname{\let\PY@bf=\textbf\def\PY@tc##1{\textcolor[rgb]{0.00,0.00,0.50}{##1}}}
\expandafter\def\csname PY@tok@go\endcsname{\def\PY@tc##1{\textcolor[rgb]{0.53,0.53,0.53}{##1}}}
\expandafter\def\csname PY@tok@gt\endcsname{\def\PY@tc##1{\textcolor[rgb]{0.00,0.27,0.87}{##1}}}
\expandafter\def\csname PY@tok@err\endcsname{\def\PY@bc##1{\setlength{\fboxsep}{0pt}\fcolorbox[rgb]{1.00,0.00,0.00}{1,1,1}{\strut ##1}}}
\expandafter\def\csname PY@tok@kc\endcsname{\let\PY@bf=\textbf\def\PY@tc##1{\textcolor[rgb]{0.00,0.50,0.00}{##1}}}
\expandafter\def\csname PY@tok@kd\endcsname{\let\PY@bf=\textbf\def\PY@tc##1{\textcolor[rgb]{0.00,0.50,0.00}{##1}}}
\expandafter\def\csname PY@tok@kn\endcsname{\let\PY@bf=\textbf\def\PY@tc##1{\textcolor[rgb]{0.00,0.50,0.00}{##1}}}
\expandafter\def\csname PY@tok@kr\endcsname{\let\PY@bf=\textbf\def\PY@tc##1{\textcolor[rgb]{0.00,0.50,0.00}{##1}}}
\expandafter\def\csname PY@tok@bp\endcsname{\def\PY@tc##1{\textcolor[rgb]{0.00,0.50,0.00}{##1}}}
\expandafter\def\csname PY@tok@fm\endcsname{\def\PY@tc##1{\textcolor[rgb]{0.00,0.00,1.00}{##1}}}
\expandafter\def\csname PY@tok@vc\endcsname{\def\PY@tc##1{\textcolor[rgb]{0.10,0.09,0.49}{##1}}}
\expandafter\def\csname PY@tok@vg\endcsname{\def\PY@tc##1{\textcolor[rgb]{0.10,0.09,0.49}{##1}}}
\expandafter\def\csname PY@tok@vi\endcsname{\def\PY@tc##1{\textcolor[rgb]{0.10,0.09,0.49}{##1}}}
\expandafter\def\csname PY@tok@vm\endcsname{\def\PY@tc##1{\textcolor[rgb]{0.10,0.09,0.49}{##1}}}
\expandafter\def\csname PY@tok@sa\endcsname{\def\PY@tc##1{\textcolor[rgb]{0.73,0.13,0.13}{##1}}}
\expandafter\def\csname PY@tok@sb\endcsname{\def\PY@tc##1{\textcolor[rgb]{0.73,0.13,0.13}{##1}}}
\expandafter\def\csname PY@tok@sc\endcsname{\def\PY@tc##1{\textcolor[rgb]{0.73,0.13,0.13}{##1}}}
\expandafter\def\csname PY@tok@dl\endcsname{\def\PY@tc##1{\textcolor[rgb]{0.73,0.13,0.13}{##1}}}
\expandafter\def\csname PY@tok@s2\endcsname{\def\PY@tc##1{\textcolor[rgb]{0.73,0.13,0.13}{##1}}}
\expandafter\def\csname PY@tok@sh\endcsname{\def\PY@tc##1{\textcolor[rgb]{0.73,0.13,0.13}{##1}}}
\expandafter\def\csname PY@tok@s1\endcsname{\def\PY@tc##1{\textcolor[rgb]{0.73,0.13,0.13}{##1}}}
\expandafter\def\csname PY@tok@mb\endcsname{\def\PY@tc##1{\textcolor[rgb]{0.40,0.40,0.40}{##1}}}
\expandafter\def\csname PY@tok@mf\endcsname{\def\PY@tc##1{\textcolor[rgb]{0.40,0.40,0.40}{##1}}}
\expandafter\def\csname PY@tok@mh\endcsname{\def\PY@tc##1{\textcolor[rgb]{0.40,0.40,0.40}{##1}}}
\expandafter\def\csname PY@tok@mi\endcsname{\def\PY@tc##1{\textcolor[rgb]{0.40,0.40,0.40}{##1}}}
\expandafter\def\csname PY@tok@il\endcsname{\def\PY@tc##1{\textcolor[rgb]{0.40,0.40,0.40}{##1}}}
\expandafter\def\csname PY@tok@mo\endcsname{\def\PY@tc##1{\textcolor[rgb]{0.40,0.40,0.40}{##1}}}
\expandafter\def\csname PY@tok@ch\endcsname{\let\PY@it=\textit\def\PY@tc##1{\textcolor[rgb]{0.25,0.50,0.50}{##1}}}
\expandafter\def\csname PY@tok@cm\endcsname{\let\PY@it=\textit\def\PY@tc##1{\textcolor[rgb]{0.25,0.50,0.50}{##1}}}
\expandafter\def\csname PY@tok@cpf\endcsname{\let\PY@it=\textit\def\PY@tc##1{\textcolor[rgb]{0.25,0.50,0.50}{##1}}}
\expandafter\def\csname PY@tok@c1\endcsname{\let\PY@it=\textit\def\PY@tc##1{\textcolor[rgb]{0.25,0.50,0.50}{##1}}}
\expandafter\def\csname PY@tok@cs\endcsname{\let\PY@it=\textit\def\PY@tc##1{\textcolor[rgb]{0.25,0.50,0.50}{##1}}}

\def\PYZbs{\char`\\}
\def\PYZus{\char`\_}
\def\PYZob{\char`\{}
\def\PYZcb{\char`\}}
\def\PYZca{\char`\^}
\def\PYZam{\char`\&}
\def\PYZlt{\char`\<}
\def\PYZgt{\char`\>}
\def\PYZsh{\char`\#}
\def\PYZpc{\char`\%}
\def\PYZdl{\char`\$}
\def\PYZhy{\char`\-}
\def\PYZsq{\char`\'}
\def\PYZdq{\char`\"}
\def\PYZti{\char`\~}
% for compatibility with earlier versions
\def\PYZat{@}
\def\PYZlb{[}
\def\PYZrb{]}
\makeatother


    % Exact colors from NB
    \definecolor{incolor}{rgb}{0.0, 0.0, 0.5}
    \definecolor{outcolor}{rgb}{0.545, 0.0, 0.0}



    
    % Prevent overflowing lines due to hard-to-break entities
    \sloppy 
    % Setup hyperref package
    \hypersetup{
      breaklinks=true,  % so long urls are correctly broken across lines
      colorlinks=true,
      urlcolor=urlcolor,
      linkcolor=linkcolor,
      citecolor=citecolor,
      }
    % Slightly bigger margins than the latex defaults
    
    \geometry{verbose,tmargin=1in,bmargin=1in,lmargin=1in,rmargin=1in}
    
    

    \begin{document}
    
    
    \maketitle
    
    

    
    \hypertarget{gestiuxf3n-y-uso-de-metadatos}{%
\section{Gestión y uso de
Metadatos}\label{gestiuxf3n-y-uso-de-metadatos}}

    \hypertarget{libreruxedas-necesarias}{%
\subsection{Librerías Necesarias}\label{libreruxedas-necesarias}}

    xml.etree.ElementTree

requests

    \begin{Verbatim}[commandchars=\\\{\}]
{\color{incolor}In [{\color{incolor}1}]:} \PY{k+kn}{import} \PY{n+nn}{xml}\PY{n+nn}{.}\PY{n+nn}{etree}\PY{n+nn}{.}\PY{n+nn}{ElementTree} \PY{k}{as} \PY{n+nn}{ET}
        \PY{k+kn}{import} \PY{n+nn}{requests}
\end{Verbatim}


    \hypertarget{metadata-attachment}{%
\subsection{Metadata attachment}\label{metadata-attachment}}

    En este Notebook descubriremos cómo pueden explotarse metadatos
publicados en formatos basados en etiquetas, como XML.

    

    Vamos a empezar por describir un par de objetos, empezando por un
cuadro, ``El grito'', de Edvard Munch.

\begin{itemize}
\tightlist
\item
  Title: El grito
\item
  Creator: Edvard Munch
\item
  Subject: ~Cuadro
\item
  Description: Cuadro al óleo de un hombre gritando
\item
  Publisher: Galeria nacional de Noruega
\item
  Contributor: -
\item
  Date: Agosto 2006
\item
  Type: óleo
\item
  Format: Lienzo, medidas
\item
  Identifier: GNN-12312
\item
  Source: -
\item
  Language: -
\item
  Relation: -
\item
  Coverage: -
\item
  Rights: Entrada a la galería
\end{itemize}

    Con Dublin Core también podemos describir datasets científicos. Vamos a
probar con:

https://zenodo.org/record/3372754\#.XcFkhE9Kg5k

\textbf{Consejo}

En el propio repositorio puedes encontrar metadatos

\begin{itemize}
\tightlist
\item
  Title:
\item
  Creator:
\item
  Subject: ~
\item
  Description:
\item
  Publisher:
\item
  Contributor:
\item
  Date:
\item
  Type:
\item
  Format:
\item
  Identifier:
\item
  Source:
\item
  Language:
\item
  Relation:
\item
  Coverage: ~
\item
  Rights:
\end{itemize}

    A partir de las descripciones, podemos crear documentos XML que sean
interpretables por máquinas (entendiendo máquinas como scripts,
software, etc).

El grito:

\begin{Shaded}
\begin{Highlighting}[]
\KeywordTok{<dc:contributor></dc:contributor>}
\KeywordTok{<dc:coverage></dc:coverage>}
\KeywordTok{<dc:creator></dc:creator>}
\KeywordTok{<dc:date>}\NormalTok{Agosto 2006}\KeywordTok{</dc:date>}
\KeywordTok{<dc:description></dc:description>}
\KeywordTok{<dc:format>}\NormalTok{Lienzo}\KeywordTok{</dc:format>}
\KeywordTok{<dc:identifier></dc:identifier>}
\KeywordTok{<dc:language></dc:language>}
\KeywordTok{<dc:publisher>}\NormalTok{Galeria Nacional de Noruega}\KeywordTok{</dc:publisher>}
\KeywordTok{<dc:relation></dc:relation>}
\KeywordTok{<dc:rights></dc:rights>}
\KeywordTok{<dc:source></dc:source>}
\KeywordTok{<dc:subject></dc:subject>}
\KeywordTok{<dc:title>}\NormalTok{El grito}\KeywordTok{</dc:title>}
\KeywordTok{<dc:type>}\NormalTok{Óleo}\KeywordTok{</dc:type>}
\end{Highlighting}
\end{Shaded}

Dataset:
\texttt{XML\ \ \textless{}dc:contributor\textgreater{}\ \textless{}/dc:contributor\textgreater{}\ \ \ \textless{}dc:coverage\textgreater{}\ \textless{}/dc:coverage\textgreater{}\ \ \ \textless{}dc:creator\textgreater{}\textless{}/dc:creator\textgreater{}\ \ \ \textless{}dc:date\textgreater{}\textless{}/dc:date\textgreater{}\ \ \ \textless{}dc:subject\textgreater{}\textless{}/dc:subject\textgreater{}\ \ \ \textless{}dc:description\textgreater{}\textless{}/dc:description\textgreater{}\ \ \ \textless{}dc:format\textgreater{}\ \ \textless{}/dc:format\textgreater{}\ \ \ \textless{}dc:identifier\textgreater{}\textless{}/dc:identifier\textgreater{}\ \ \ \textless{}dc:language\textgreater{}\ \textless{}/dc:language\textgreater{}\ \ \ \textless{}dc:publisher\textgreater{}\textless{}/dc:publisher\textgreater{}\ \ \ \textless{}dc:relation\textgreater{}\ \textless{}/dc:relation\textgreater{}\ \ \ \textless{}dc:rights\textgreater{}\ \textless{}/dc:rights\textgreater{}\ \ \ \textless{}dc:source\textgreater{}\ \textless{}/dc:source\textgreater{}\ \ \ \textless{}dc:title\textgreater{}\textless{}/dc:title\textgreater{}\ \ \ \textless{}dc:type\textgreater{}\textless{}/dc:type\textgreater{}}

    Ahora vamos a ver cómo podemos manejar estos datos en Python. Para ello,
utilizaremos la librería xml.

Para crear un documento XML bien formado, es necesario definir dónde
está descrito el prefijo Dublin Core o ``dc:''. Para ello, añadimos
antes de los datos la siguiente cabecera:

\begin{Shaded}
\begin{Highlighting}[]
\KeywordTok{<?xml}\NormalTok{ version="1.0" encoding="UTF-8" standalone="no"}\KeywordTok{?><?xml-stylesheet}\NormalTok{ type="text/xsl" href="/webservices/catalog/xsl/searchRetrieveResponse.xsl"}\KeywordTok{?>}
\KeywordTok{<searchRetrieveResponse}\OtherTok{ xmlns:oclcterms=}\StringTok{"http://purl.org/oclc/terms/"}\OtherTok{ xmlns:dc=}\StringTok{"http://purl.org/dc/elements/1.1/"}\OtherTok{ xmlns:diag=}\StringTok{"http://www.loc.gov/zing/srw/diagnostic/"}\OtherTok{ xmlns:xsi=}\StringTok{"http://www.w3.org/2001/XMLSchema-instance"}\KeywordTok{>}
\end{Highlighting}
\end{Shaded}

Sin olvidar añadir al final:

\begin{Shaded}
\begin{Highlighting}[]
\ErrorTok{<}\NormalTok{/searchRetrieveResponse>}
\end{Highlighting}
\end{Shaded}

    \begin{Verbatim}[commandchars=\\\{\}]
{\color{incolor}In [{\color{incolor}2}]:} \PY{k+kn}{import} \PY{n+nn}{xml}\PY{n+nn}{.}\PY{n+nn}{etree}\PY{n+nn}{.}\PY{n+nn}{ElementTree} \PY{k}{as} \PY{n+nn}{ET}
        \PY{n}{dc\PYZus{}xml} \PY{o}{=} \PY{l+s+s1}{\PYZsq{}\PYZsq{}\PYZsq{}}\PY{l+s+s1}{\PYZlt{}?xml version=}\PY{l+s+s1}{\PYZdq{}}\PY{l+s+s1}{1.0}\PY{l+s+s1}{\PYZdq{}}\PY{l+s+s1}{ encoding=}\PY{l+s+s1}{\PYZdq{}}\PY{l+s+s1}{UTF\PYZhy{}8}\PY{l+s+s1}{\PYZdq{}}\PY{l+s+s1}{ standalone=}\PY{l+s+s1}{\PYZdq{}}\PY{l+s+s1}{no}\PY{l+s+s1}{\PYZdq{}}\PY{l+s+s1}{?\PYZgt{}\PYZlt{}?xml\PYZhy{}stylesheet type=}\PY{l+s+s1}{\PYZdq{}}\PY{l+s+s1}{text/xsl}\PY{l+s+s1}{\PYZdq{}}\PY{l+s+s1}{ href=}\PY{l+s+s1}{\PYZdq{}}\PY{l+s+s1}{/webservices/catalog/xsl/searchRetrieveResponse.xsl}\PY{l+s+s1}{\PYZdq{}}\PY{l+s+s1}{?\PYZgt{}}
        \PY{l+s+s1}{\PYZlt{}searchRetrieveResponse xmlns:oclcterms=}\PY{l+s+s1}{\PYZdq{}}\PY{l+s+s1}{http://purl.org/oclc/terms/}\PY{l+s+s1}{\PYZdq{}}\PY{l+s+s1}{ xmlns:dc=}\PY{l+s+s1}{\PYZdq{}}\PY{l+s+s1}{http://purl.org/dc/elements/1.1/}\PY{l+s+s1}{\PYZdq{}}\PY{l+s+s1}{ xmlns:diag=}\PY{l+s+s1}{\PYZdq{}}\PY{l+s+s1}{http://www.loc.gov/zing/srw/diagnostic/}\PY{l+s+s1}{\PYZdq{}}\PY{l+s+s1}{ xmlns:xsi=}\PY{l+s+s1}{\PYZdq{}}\PY{l+s+s1}{http://www.w3.org/2001/XMLSchema\PYZhy{}instance}\PY{l+s+s1}{\PYZdq{}}\PY{l+s+s1}{\PYZgt{}}
        
        
        \PY{l+s+s1}{     \PYZlt{}dc:contributor\PYZgt{}Edvard Munch \PYZlt{}/dc:contributor\PYZgt{}}
        \PY{l+s+s1}{  \PYZlt{}dc:coverage\PYZgt{}Lugar indeterminado\PYZlt{}/dc:coverage\PYZgt{}}
        \PY{l+s+s1}{  \PYZlt{}dc:creator\PYZgt{}Edvard Munch \PYZlt{}/dc:creator\PYZgt{}}
        \PY{l+s+s1}{  \PYZlt{}dc:date\PYZgt{}1910\PYZlt{}/dc:date\PYZgt{}}
        \PY{l+s+s1}{  \PYZlt{}dc:description\PYZgt{}Cuadro...\PYZlt{}/dc:description\PYZgt{}}
        \PY{l+s+s1}{  \PYZlt{}dc:format\PYZgt{}Oleo sobre carton\PYZlt{}/dc:format\PYZgt{}}
        \PY{l+s+s1}{  \PYZlt{}dc:identifier\PYZgt{}id\PYZus{}museo\PYZus{}grito\PYZlt{}/dc:identifier\PYZgt{}}
        \PY{l+s+s1}{  \PYZlt{}dc:language\PYZgt{}\PYZlt{}/dc:language\PYZgt{}}
        \PY{l+s+s1}{  \PYZlt{}dc:publisher\PYZgt{}Galeria nacional de Oslo\PYZlt{}/dc:publisher\PYZgt{}}
        \PY{l+s+s1}{  \PYZlt{}dc:relation\PYZgt{}cuadro1, cuadro2, cuadro3\PYZlt{}/dc:relation\PYZgt{}}
        \PY{l+s+s1}{  \PYZlt{}dc:rights\PYZgt{}Acceso al museo\PYZlt{}/dc:rights\PYZgt{}}
        \PY{l+s+s1}{  \PYZlt{}dc:source\PYZgt{}\PYZlt{}/dc:source\PYZgt{}}
        \PY{l+s+s1}{  \PYZlt{}dc:title\PYZgt{}El gripo\PYZlt{}/dc:title\PYZgt{}}
        \PY{l+s+s1}{  \PYZlt{}dc:type\PYZgt{}Cuadro\PYZlt{}/dc:type\PYZgt{}}
        
        
        
        \PY{l+s+s1}{\PYZlt{}/searchRetrieveResponse\PYZgt{}}\PY{l+s+s1}{\PYZsq{}\PYZsq{}\PYZsq{}}
        
        \PY{n}{tree} \PY{o}{=} \PY{n}{ET}\PY{o}{.}\PY{n}{fromstring}\PY{p}{(}\PY{n}{dc\PYZus{}xml}\PY{p}{)}
        \PY{n}{tree}
\end{Verbatim}


\begin{Verbatim}[commandchars=\\\{\}]
{\color{outcolor}Out[{\color{outcolor}2}]:} <Element 'searchRetrieveResponse' at 0x7ff932dbae58>
\end{Verbatim}
            
    Si queremos recorrer los elementos del XML que hemos formado, podemos
utilizar un bucle, teniendo en cuenta que la información que nos
interesa la tenemos en `searchRetrieveResponse':

    \begin{Verbatim}[commandchars=\\\{\}]
{\color{incolor}In [{\color{incolor}6}]:} \PY{k}{for} \PY{n}{table} \PY{o+ow}{in} \PY{n}{tree}\PY{o}{.}\PY{n}{getiterator}\PY{p}{(}\PY{l+s+s1}{\PYZsq{}}\PY{l+s+s1}{searchRetrieveResponse}\PY{l+s+s1}{\PYZsq{}}\PY{p}{)}\PY{p}{:} \PY{c+c1}{\PYZsh{}Se genera un iterador partir de la raíz del árbol}
            \PY{k}{for} \PY{n}{child} \PY{o+ow}{in} \PY{n}{table}\PY{p}{:}
                \PY{n+nb}{print}\PY{p}{(}\PY{n}{child}\PY{o}{.}\PY{n}{tag}\PY{p}{,} \PY{n}{child}\PY{o}{.}\PY{n}{text}\PY{p}{)}
\end{Verbatim}


    \begin{Verbatim}[commandchars=\\\{\}]
\{http://purl.org/dc/elements/1.1/\}contributor Edvard Munch 
\{http://purl.org/dc/elements/1.1/\}coverage Lugar indeterminado
\{http://purl.org/dc/elements/1.1/\}creator Edvard Munch 
\{http://purl.org/dc/elements/1.1/\}date 1910
\{http://purl.org/dc/elements/1.1/\}description Cuadro{\ldots}
\{http://purl.org/dc/elements/1.1/\}format Oleo sobre carton
\{http://purl.org/dc/elements/1.1/\}identifier id\_museo\_grito
\{http://purl.org/dc/elements/1.1/\}language None
\{http://purl.org/dc/elements/1.1/\}publisher Galeria nacional de Oslo
\{http://purl.org/dc/elements/1.1/\}relation cuadro1, cuadro2, cuadro3
\{http://purl.org/dc/elements/1.1/\}rights Acceso al museo
\{http://purl.org/dc/elements/1.1/\}source None
\{http://purl.org/dc/elements/1.1/\}title El gripo
\{http://purl.org/dc/elements/1.1/\}type Cuadro

    \end{Verbatim}

    Observa que, al utilizar el prefijo `dc:' e indicarle que está descrito
en la URL `http://purl.org/dc/elements/1.1/', la eqtiqueta o ``tag''
aparece como, por ejemplo \{URL\}contributor.

Prueba a mostrar los metadatos que has creado a partir del cuadro y del
dataset:

    \begin{Verbatim}[commandchars=\\\{\}]
{\color{incolor}In [{\color{incolor}7}]:} \PY{k}{for} \PY{n}{table} \PY{o+ow}{in} \PY{n}{tree}\PY{o}{.}\PY{n}{getiterator}\PY{p}{(}\PY{l+s+s1}{\PYZsq{}}\PY{l+s+s1}{searchRetrieveResponse}\PY{l+s+s1}{\PYZsq{}}\PY{p}{)}\PY{p}{:}
            \PY{k}{for} \PY{n}{child} \PY{o+ow}{in} \PY{n}{table}\PY{p}{:}
                \PY{n+nb}{print}\PY{p}{(}\PY{n}{child}\PY{o}{.}\PY{n}{text}\PY{p}{)}
\end{Verbatim}


    \begin{Verbatim}[commandchars=\\\{\}]
Edvard Munch 
Lugar indeterminado
Edvard Munch 
1910
Cuadro{\ldots}
Oleo sobre carton
id\_museo\_grito
None
Galeria nacional de Oslo
cuadro1, cuadro2, cuadro3
Acceso al museo
None
El gripo
Cuadro

    \end{Verbatim}

    Utilizando findall() sobre el arbol (tree), podemos encontrar todos los
elementos con una etiqueta determinada.

    \begin{Verbatim}[commandchars=\\\{\}]
{\color{incolor}In [{\color{incolor}8}]:} \PY{n}{relation} \PY{o}{=} \PY{n}{tree}\PY{o}{.}\PY{n}{findall}\PY{p}{(}\PY{l+s+s1}{\PYZsq{}}\PY{l+s+s1}{\PYZob{}}\PY{l+s+s1}{http://purl.org/dc/elements/1.1/\PYZcb{}relation}\PY{l+s+s1}{\PYZsq{}}\PY{p}{)}
        \PY{n+nb}{print}\PY{p}{(}\PY{n}{relation}\PY{p}{)}
\end{Verbatim}


    \begin{Verbatim}[commandchars=\\\{\}]
[<Element '\{http://purl.org/dc/elements/1.1/\}relation' at 0x7f923c9d55e8>]

    \end{Verbatim}

    Ten en cuenta que lo que encontramos es, en realidad, una parte del
documento XML, por lo que hay que iterarlo como antes:

    \begin{Verbatim}[commandchars=\\\{\}]
{\color{incolor}In [{\color{incolor}9}]:} \PY{k}{for} \PY{n}{child} \PY{o+ow}{in} \PY{n}{relation}\PY{p}{:}
            \PY{n+nb}{print}\PY{p}{(}\PY{n}{child}\PY{o}{.}\PY{n}{tag}\PY{p}{,} \PY{n}{child}\PY{o}{.}\PY{n}{text}\PY{p}{)}
\end{Verbatim}


    \begin{Verbatim}[commandchars=\\\{\}]
\{http://purl.org/dc/elements/1.1/\}relation cuadro1, cuadro2, cuadro3

    \end{Verbatim}

    XML utiliza prefijos para no necesitar referenciar a la URL de un tipo
cada vez, lo podemos ver en la cabecera:

\begin{Shaded}
\begin{Highlighting}[]
\KeywordTok{<?xml}\NormalTok{ version="1.0" encoding="UTF-8" standalone="no"}\KeywordTok{?><?xml-stylesheet}\NormalTok{ type="text/xsl" href="/webservices/catalog/xsl/searchRetrieveResponse.xsl"}\KeywordTok{?>}
\KeywordTok{<searchRetrieveResponse}\OtherTok{ xmlns:oclcterms=}\StringTok{"http://purl.org/oclc/terms/"}\OtherTok{ xmlns:dc=}\StringTok{"http://purl.org/dc/elements/1.1/"}\OtherTok{ xmlns:diag=}\StringTok{"http://www.loc.gov/zing/srw/diagnostic/"}\OtherTok{ xmlns:xsi=}\StringTok{"http://www.w3.org/2001/XMLSchema-instance"}\KeywordTok{>}\NormalTok{'''}
\end{Highlighting}
\end{Shaded}

Por ejemplo, cada vez que queremos utilizar un tipo de Dublin Core,
utilizamos el prefijo dc: que equivale a llamar a la definición:

xmlns:dc=``http://purl.org/dc/elements/1.1/''

Sin embargo, para utilizar ElementTree en Python, tenemos que utilizar
la URL completa. Esto puede resultar un poco engorroso, así que podemos
definir el namespace para utilizar también el prefijo:

    \begin{Verbatim}[commandchars=\\\{\}]
{\color{incolor}In [{\color{incolor}10}]:} \PY{n}{namespaces} \PY{o}{=} \PY{p}{\PYZob{}}\PY{l+s+s1}{\PYZsq{}}\PY{l+s+s1}{dc}\PY{l+s+s1}{\PYZsq{}}\PY{p}{:} \PY{l+s+s1}{\PYZsq{}}\PY{l+s+s1}{http://purl.org/dc/elements/1.1/}\PY{l+s+s1}{\PYZsq{}}\PY{p}{\PYZcb{}} \PY{c+c1}{\PYZsh{} add more as needed}
         
         \PY{n}{tree}\PY{o}{.}\PY{n}{find}\PY{p}{(}\PY{l+s+s1}{\PYZsq{}}\PY{l+s+s1}{dc:rights}\PY{l+s+s1}{\PYZsq{}}\PY{p}{,}\PY{n}{namespaces}\PY{p}{)}\PY{o}{.}\PY{n}{text}
\end{Verbatim}


\begin{Verbatim}[commandchars=\\\{\}]
{\color{outcolor}Out[{\color{outcolor}10}]:} 'Acceso al museo'
\end{Verbatim}
            
    Los documentos XML, aparte de las etiquetas y los valores, pueden
contener atributos. Dado el siguiente ejemplo, vamos a ver cómo obtener
la lista y los valores de los atributos

    \begin{Verbatim}[commandchars=\\\{\}]
{\color{incolor}In [{\color{incolor}13}]:} \PY{n}{dc\PYZus{}xml} \PY{o}{=} \PY{l+s+s1}{\PYZsq{}\PYZsq{}\PYZsq{}}\PY{l+s+s1}{\PYZlt{}?xml version=}\PY{l+s+s1}{\PYZdq{}}\PY{l+s+s1}{1.0}\PY{l+s+s1}{\PYZdq{}}\PY{l+s+s1}{ encoding=}\PY{l+s+s1}{\PYZdq{}}\PY{l+s+s1}{UTF\PYZhy{}8}\PY{l+s+s1}{\PYZdq{}}\PY{l+s+s1}{ standalone=}\PY{l+s+s1}{\PYZdq{}}\PY{l+s+s1}{no}\PY{l+s+s1}{\PYZdq{}}\PY{l+s+s1}{?\PYZgt{}\PYZlt{}?xml\PYZhy{}stylesheet type=}\PY{l+s+s1}{\PYZdq{}}\PY{l+s+s1}{text/xsl}\PY{l+s+s1}{\PYZdq{}}\PY{l+s+s1}{ href=}\PY{l+s+s1}{\PYZdq{}}\PY{l+s+s1}{/webservices/catalog/xsl/searchRetrieveResponse.xsl}\PY{l+s+s1}{\PYZdq{}}\PY{l+s+s1}{?\PYZgt{}}
         \PY{l+s+s1}{\PYZlt{}searchRetrieveResponse xmlns:oclcterms=}\PY{l+s+s1}{\PYZdq{}}\PY{l+s+s1}{http://purl.org/oclc/terms/}\PY{l+s+s1}{\PYZdq{}}\PY{l+s+s1}{ xmlns:dc=}\PY{l+s+s1}{\PYZdq{}}\PY{l+s+s1}{http://purl.org/dc/elements/1.1/}\PY{l+s+s1}{\PYZdq{}}\PY{l+s+s1}{ xmlns:diag=}\PY{l+s+s1}{\PYZdq{}}\PY{l+s+s1}{http://www.loc.gov/zing/srw/diagnostic/}\PY{l+s+s1}{\PYZdq{}}\PY{l+s+s1}{ xmlns:xsi=}\PY{l+s+s1}{\PYZdq{}}\PY{l+s+s1}{http://www.w3.org/2001/XMLSchema\PYZhy{}instance}\PY{l+s+s1}{\PYZdq{}}\PY{l+s+s1}{\PYZgt{}}
         \PY{l+s+s1}{\PYZlt{}dc:contributor\PYZgt{}asdsadsad\PYZlt{}/dc:contributor\PYZgt{}}
         \PY{l+s+s1}{\PYZlt{}dc:coverage\PYZgt{}dfsd\PYZlt{}/dc:coverage\PYZgt{}}
         \PY{l+s+s1}{\PYZlt{}dc:creator\PYZgt{}sadsa\PYZlt{}/dc:creator\PYZgt{}}
         \PY{l+s+s1}{\PYZlt{}dc:date\PYZgt{}sadas\PYZlt{}/dc:date\PYZgt{}}
         \PY{l+s+s1}{\PYZlt{}dc:description atributo1=}\PY{l+s+s1}{\PYZdq{}}\PY{l+s+s1}{valor1}\PY{l+s+s1}{\PYZdq{}}\PY{l+s+s1}{ atributo2=}\PY{l+s+s1}{\PYZdq{}}\PY{l+s+s1}{valor2}\PY{l+s+s1}{\PYZdq{}}\PY{l+s+s1}{ lang=}\PY{l+s+s1}{\PYZdq{}}\PY{l+s+s1}{ES}\PY{l+s+s1}{\PYZdq{}}\PY{l+s+s1}{\PYZgt{}sadsa\PYZlt{}/dc:description\PYZgt{}}
         \PY{l+s+s1}{\PYZlt{}dc:format\PYZgt{}sadasd\PYZlt{}/dc:format\PYZgt{}}
         \PY{l+s+s1}{\PYZlt{}dc:identifier\PYZgt{}sadsad\PYZlt{}/dc:identifier\PYZgt{}}
         \PY{l+s+s1}{\PYZlt{}dc:language\PYZgt{}asdasd\PYZlt{}/dc:language\PYZgt{}}
         \PY{l+s+s1}{\PYZlt{}dc:publisher\PYZgt{}wqewq\PYZlt{}/dc:publisher\PYZgt{}}
         \PY{l+s+s1}{\PYZlt{}dc:relation \PYZgt{}wqeqw\PYZlt{}/dc:relation\PYZgt{}}
         \PY{l+s+s1}{\PYZlt{}dc:rights\PYZgt{}ffefe\PYZlt{}/dc:rights\PYZgt{}}
         \PY{l+s+s1}{\PYZlt{}dc:source\PYZgt{}vfvf\PYZlt{}/dc:source\PYZgt{}}
         \PY{l+s+s1}{\PYZlt{}dc:title\PYZgt{}wqewqe\PYZlt{}/dc:title\PYZgt{}}
         \PY{l+s+s1}{\PYZlt{}dc:type\PYZgt{}ewfrb\PYZlt{}/dc:type\PYZgt{}}
         \PY{l+s+s1}{\PYZlt{}/searchRetrieveResponse\PYZgt{}}\PY{l+s+s1}{\PYZsq{}\PYZsq{}\PYZsq{}}
         
         \PY{n}{tree2} \PY{o}{=} \PY{n}{ET}\PY{o}{.}\PY{n}{fromstring}\PY{p}{(}\PY{n}{dc\PYZus{}xml}\PY{p}{)}
\end{Verbatim}


    \begin{Verbatim}[commandchars=\\\{\}]
{\color{incolor}In [{\color{incolor}14}]:} \PY{n}{tree2}\PY{o}{.}\PY{n}{find}\PY{p}{(}\PY{l+s+s1}{\PYZsq{}}\PY{l+s+s1}{dc:description}\PY{l+s+s1}{\PYZsq{}}\PY{p}{,}\PY{n}{namespaces}\PY{p}{)}\PY{o}{.}\PY{n}{attrib}
\end{Verbatim}


\begin{Verbatim}[commandchars=\\\{\}]
{\color{outcolor}Out[{\color{outcolor}14}]:} \{'atributo1': 'valor1', 'atributo2': 'valor2', 'lang': 'ES'\}
\end{Verbatim}
            
    Conociendo los nombres de estos atributos, puedes extraer su valor. Esto
serviría para dar una información adicional al contenido de la etiqueta.
Por ejemplo, se podría añadir el idioma como atributo en la descripción.

    \begin{Verbatim}[commandchars=\\\{\}]
{\color{incolor}In [{\color{incolor}15}]:} \PY{n+nb}{print}\PY{p}{(}\PY{n}{tree2}\PY{o}{.}\PY{n}{find}\PY{p}{(}\PY{l+s+s1}{\PYZsq{}}\PY{l+s+s1}{dc:description}\PY{l+s+s1}{\PYZsq{}}\PY{p}{,}\PY{n}{namespaces}\PY{p}{)}\PY{o}{.}\PY{n}{attrib}\PY{p}{[}\PY{l+s+s1}{\PYZsq{}}\PY{l+s+s1}{atributo1}\PY{l+s+s1}{\PYZsq{}}\PY{p}{]}\PY{p}{)}
         \PY{n+nb}{print}\PY{p}{(}\PY{n}{tree2}\PY{o}{.}\PY{n}{find}\PY{p}{(}\PY{l+s+s1}{\PYZsq{}}\PY{l+s+s1}{dc:description}\PY{l+s+s1}{\PYZsq{}}\PY{p}{,}\PY{n}{namespaces}\PY{p}{)}\PY{o}{.}\PY{n}{attrib}\PY{p}{[}\PY{l+s+s1}{\PYZsq{}}\PY{l+s+s1}{atributo2}\PY{l+s+s1}{\PYZsq{}}\PY{p}{]}\PY{p}{)}
\end{Verbatim}


    \begin{Verbatim}[commandchars=\\\{\}]
valor1
valor2

    \end{Verbatim}

    Vamos a analizar un documento XML más complejo, empezando por
descargarlo:

    \begin{Verbatim}[commandchars=\\\{\}]
{\color{incolor}In [{\color{incolor}57}]:} \PY{k+kn}{import} \PY{n+nn}{requests}
         
         \PY{n}{response} \PY{o}{=} \PY{n}{requests}\PY{o}{.}\PY{n}{get}\PY{p}{(}\PY{l+s+s1}{\PYZsq{}}\PY{l+s+s1}{https://gist.githubusercontent.com/vivien/580729/raw/651d1b216357c0d7d9fc47075071fb482e11fb36/dublincore\PYZhy{}example.xml}\PY{l+s+s1}{\PYZsq{}}\PY{p}{)}
         \PY{k}{if} \PY{n}{response}\PY{o}{.}\PY{n}{status\PYZus{}code} \PY{o}{==} \PY{l+m+mi}{200}\PY{p}{:}
             \PY{k}{with} \PY{n+nb}{open}\PY{p}{(}\PY{l+s+s2}{\PYZdq{}}\PY{l+s+s2}{./dublincore\PYZhy{}example.xml}\PY{l+s+s2}{\PYZdq{}}\PY{p}{,} \PY{l+s+s1}{\PYZsq{}}\PY{l+s+s1}{wb}\PY{l+s+s1}{\PYZsq{}}\PY{p}{)} \PY{k}{as} \PY{n}{f}\PY{p}{:}
                 \PY{n}{f}\PY{o}{.}\PY{n}{write}\PY{p}{(}\PY{n}{response}\PY{o}{.}\PY{n}{content}\PY{p}{)}
\end{Verbatim}


    \textbf{Recuerda!}

Jupyter permite ejecutar ciertos comandos bash

    \begin{Verbatim}[commandchars=\\\{\}]
{\color{incolor}In [{\color{incolor}17}]:} \PY{n}{ls}
\end{Verbatim}


    \begin{Verbatim}[commandchars=\\\{\}]
dublincore-example.xml  metadataIntro.ipynb

    \end{Verbatim}

    Y lo cargamos en python:

    \begin{Verbatim}[commandchars=\\\{\}]
{\color{incolor}In [{\color{incolor}58}]:} \PY{n}{tree} \PY{o}{=} \PY{n}{ET}\PY{o}{.}\PY{n}{parse}\PY{p}{(}\PY{l+s+s1}{\PYZsq{}}\PY{l+s+s1}{dublincore\PYZhy{}example.xml}\PY{l+s+s1}{\PYZsq{}}\PY{p}{)}
         \PY{n}{namespaces} \PY{o}{=} \PY{p}{\PYZob{}}\PY{l+s+s1}{\PYZsq{}}\PY{l+s+s1}{dc}\PY{l+s+s1}{\PYZsq{}}\PY{p}{:} \PY{l+s+s1}{\PYZsq{}}\PY{l+s+s1}{http://purl.org/dc/elements/1.1/}\PY{l+s+s1}{\PYZsq{}}\PY{p}{\PYZcb{}} \PY{c+c1}{\PYZsh{} add more as needed}
         \PY{k}{for} \PY{n}{table} \PY{o+ow}{in} \PY{n}{tree}\PY{o}{.}\PY{n}{getiterator}\PY{p}{(}\PY{l+s+s1}{\PYZsq{}}\PY{l+s+s1}{\PYZob{}}\PY{l+s+s1}{http://www.loc.gov/zing/srw/\PYZcb{}searchRetrieveResponse}\PY{l+s+s1}{\PYZsq{}}\PY{p}{)}\PY{p}{:}
             \PY{k}{for} \PY{n}{child} \PY{o+ow}{in} \PY{n}{table}\PY{p}{:}
                 \PY{n+nb}{print}\PY{p}{(}\PY{n}{child}\PY{o}{.}\PY{n}{tag}\PY{p}{,} \PY{n}{child}\PY{o}{.}\PY{n}{text}\PY{p}{)}
\end{Verbatim}


    \begin{Verbatim}[commandchars=\\\{\}]
\{http://www.loc.gov/zing/srw/\}version 1.1
\{http://www.loc.gov/zing/srw/\}numberOfRecords 33587
\{http://www.loc.gov/zing/srw/\}records 

\{http://www.loc.gov/zing/srw/\}nextRecordPosition 11
\{http://www.loc.gov/zing/srw/\}resultSetIdleTime None
\{http://www.loc.gov/zing/srw/\}echoedSearchRetrieveRequest 


    \end{Verbatim}

    \begin{Verbatim}[commandchars=\\\{\}]
{\color{incolor}In [{\color{incolor}59}]:} \PY{n}{all\PYZus{}records} \PY{o}{=} \PY{n}{tree}\PY{o}{.}\PY{n}{findall}\PY{p}{(}\PY{l+s+s1}{\PYZsq{}}\PY{l+s+s1}{\PYZob{}}\PY{l+s+s1}{http://www.loc.gov/zing/srw/\PYZcb{}records}\PY{l+s+s1}{\PYZsq{}}\PY{p}{)}
         \PY{n+nb}{print}\PY{p}{(}\PY{n}{all\PYZus{}records}\PY{p}{)}
\end{Verbatim}


    \begin{Verbatim}[commandchars=\\\{\}]
[<Element '\{http://www.loc.gov/zing/srw/\}records' at 0x7f923c09c228>]

    \end{Verbatim}

    \begin{Verbatim}[commandchars=\\\{\}]
{\color{incolor}In [{\color{incolor}60}]:} \PY{k}{for} \PY{n}{table} \PY{o+ow}{in} \PY{n}{tree}\PY{o}{.}\PY{n}{getiterator}\PY{p}{(}\PY{l+s+s1}{\PYZsq{}}\PY{l+s+s1}{\PYZob{}}\PY{l+s+s1}{http://www.loc.gov/zing/srw/\PYZcb{}record}\PY{l+s+s1}{\PYZsq{}}\PY{p}{)}\PY{p}{:}
             \PY{k}{for} \PY{n}{child} \PY{o+ow}{in} \PY{n}{table}\PY{p}{:}
                 \PY{n+nb}{print}\PY{p}{(}\PY{n}{child}\PY{o}{.}\PY{n}{tag}\PY{p}{,} \PY{n}{child}\PY{o}{.}\PY{n}{text}\PY{p}{)}
\end{Verbatim}


    \begin{Verbatim}[commandchars=\\\{\}]
\{http://www.loc.gov/zing/srw/\}recordSchema info:srw/schema/1/dc
\{http://www.loc.gov/zing/srw/\}recordPacking xml
\{http://www.loc.gov/zing/srw/\}recordData 

\{http://www.loc.gov/zing/srw/\}recordSchema info:srw/schema/1/dc
\{http://www.loc.gov/zing/srw/\}recordPacking xml
\{http://www.loc.gov/zing/srw/\}recordData 

\{http://www.loc.gov/zing/srw/\}recordSchema info:srw/schema/1/dc
\{http://www.loc.gov/zing/srw/\}recordPacking xml
\{http://www.loc.gov/zing/srw/\}recordData 

\{http://www.loc.gov/zing/srw/\}recordSchema info:srw/schema/1/dc
\{http://www.loc.gov/zing/srw/\}recordPacking xml
\{http://www.loc.gov/zing/srw/\}recordData 

\{http://www.loc.gov/zing/srw/\}recordSchema info:srw/schema/1/dc
\{http://www.loc.gov/zing/srw/\}recordPacking xml
\{http://www.loc.gov/zing/srw/\}recordData 

\{http://www.loc.gov/zing/srw/\}recordSchema info:srw/schema/1/dc
\{http://www.loc.gov/zing/srw/\}recordPacking xml
\{http://www.loc.gov/zing/srw/\}recordData 

\{http://www.loc.gov/zing/srw/\}recordSchema info:srw/schema/1/dc
\{http://www.loc.gov/zing/srw/\}recordPacking xml
\{http://www.loc.gov/zing/srw/\}recordData 

\{http://www.loc.gov/zing/srw/\}recordSchema info:srw/schema/1/dc
\{http://www.loc.gov/zing/srw/\}recordPacking xml
\{http://www.loc.gov/zing/srw/\}recordData 

\{http://www.loc.gov/zing/srw/\}recordSchema info:srw/schema/1/dc
\{http://www.loc.gov/zing/srw/\}recordPacking xml
\{http://www.loc.gov/zing/srw/\}recordData 

\{http://www.loc.gov/zing/srw/\}recordSchema info:srw/schema/1/dc
\{http://www.loc.gov/zing/srw/\}recordPacking xml
\{http://www.loc.gov/zing/srw/\}recordData 


    \end{Verbatim}

    \begin{Verbatim}[commandchars=\\\{\}]
{\color{incolor}In [{\color{incolor}61}]:} \PY{k}{for} \PY{n}{table} \PY{o+ow}{in} \PY{n}{tree}\PY{o}{.}\PY{n}{getiterator}\PY{p}{(}\PY{l+s+s1}{\PYZsq{}}\PY{l+s+s1}{\PYZob{}}\PY{l+s+s1}{http://www.loc.gov/zing/srw/\PYZcb{}recordData}\PY{l+s+s1}{\PYZsq{}}\PY{p}{)}\PY{p}{:}
             \PY{k}{for} \PY{n}{child} \PY{o+ow}{in} \PY{n}{table}\PY{p}{:}
                 \PY{n+nb}{print}\PY{p}{(}\PY{n}{child}\PY{o}{.}\PY{n}{tag}\PY{p}{,} \PY{n}{child}\PY{o}{.}\PY{n}{text}\PY{p}{)}
\end{Verbatim}


    \begin{Verbatim}[commandchars=\\\{\}]
\{http://www.loc.gov/zing/srw/\}oclcdcs 

\{http://www.loc.gov/zing/srw/\}oclcdcs 

\{http://www.loc.gov/zing/srw/\}oclcdcs 

\{http://www.loc.gov/zing/srw/\}oclcdcs 

\{http://www.loc.gov/zing/srw/\}oclcdcs 

\{http://www.loc.gov/zing/srw/\}oclcdcs 

\{http://www.loc.gov/zing/srw/\}oclcdcs 

\{http://www.loc.gov/zing/srw/\}oclcdcs 

\{http://www.loc.gov/zing/srw/\}oclcdcs 

\{http://www.loc.gov/zing/srw/\}oclcdcs 


    \end{Verbatim}

    \begin{Verbatim}[commandchars=\\\{\}]
{\color{incolor}In [{\color{incolor}62}]:} \PY{k}{for} \PY{n}{table} \PY{o+ow}{in} \PY{n}{tree}\PY{o}{.}\PY{n}{getiterator}\PY{p}{(}\PY{l+s+s1}{\PYZsq{}}\PY{l+s+s1}{\PYZob{}}\PY{l+s+s1}{http://www.loc.gov/zing/srw/\PYZcb{}oclcdcs}\PY{l+s+s1}{\PYZsq{}}\PY{p}{)}\PY{p}{:}
             \PY{k}{for} \PY{n}{child} \PY{o+ow}{in} \PY{n}{table}\PY{p}{:}
                 \PY{n+nb}{print}\PY{p}{(}\PY{n}{child}\PY{o}{.}\PY{n}{tag}\PY{p}{,} \PY{n}{child}\PY{o}{.}\PY{n}{text}\PY{p}{)}
\end{Verbatim}


    \begin{Verbatim}[commandchars=\\\{\}]
\{http://purl.org/dc/elements/1.1/\}creator Snelling, Lauraine.
\{http://purl.org/dc/elements/1.1/\}date c2003
\{http://purl.org/dc/elements/1.1/\}description "Ruby Torvald sets out on a daunting journey with her young sister, Opal, to hopefully see their long-lost father once more and claim the promised inheritance. But instead of the treasure they expected, the sisters discover something most shocking." -- Book Cover.
\{http://purl.org/dc/elements/1.1/\}format 320 p. ; 22 cm.
\{http://purl.org/dc/elements/1.1/\}identifier 0764290762
\{http://purl.org/dc/elements/1.1/\}identifier 9780764290763
\{http://purl.org/dc/elements/1.1/\}identifier 0764222228
\{http://purl.org/dc/elements/1.1/\}identifier 9780764222221
\{http://purl.org/dc/elements/1.1/\}language eng
\{http://purl.org/dc/elements/1.1/\}publisher Bethany House Publishers
\{http://purl.org/dc/elements/1.1/\}relation Dakotah treasures ; 1
\{http://purl.org/dc/elements/1.1/\}subject Inheritance and succession--Fiction.
\{http://purl.org/dc/elements/1.1/\}subject Fathers and daughters--Fiction.
\{http://purl.org/dc/elements/1.1/\}subject Women pioneers--Fiction.
\{http://purl.org/dc/elements/1.1/\}subject Sisters--Fiction.
\{http://purl.org/dc/elements/1.1/\}subject Medora (N.D.)--Fiction.
\{http://purl.org/dc/elements/1.1/\}title Ruby 
\{http://purl.org/dc/elements/1.1/\}type Christian fiction.
\{http://purl.org/dc/elements/1.1/\}type Western stories.
\{http://purl.org/dc/elements/1.1/\}type Text
\{http://purl.org/oclc/terms/\}recordCreationDate 030206
\{http://purl.org/oclc/terms/\}recordIdentifier   2003002571
\{http://purl.org/oclc/terms/\}recordIdentifier 51647374
\{http://purl.org/dc/elements/1.1/\}creator Creech, Sharon.
\{http://purl.org/dc/elements/1.1/\}date c2002
\{http://purl.org/dc/elements/1.1/\}description 1st ed.
\{http://purl.org/dc/elements/1.1/\}description Thirteen-year-old fraternal twins Dallas and Florida have grown up in a terrible orphanage but their lives change forever when an eccentric but sweet older couple invites them each on an adventure, beginning in an almost magical place called Ruby Holler.
\{http://purl.org/dc/elements/1.1/\}description The Silver Bird -- The Boxton Creek Home -- Ruby Holler -- Mush -- Thinking Corners -- The Opportunity -- Doubts -- Hansel and Gretel -- The God -- The Egg -- The Grump -- Work -- Gravy -- Wood -- Conversations in the Night -- The Axe -- The Rocker -- The Trepids -- Understone Funds -- Through the Holler -- Lost and Found -- A Trip to Boxton -- Ready -- Tiller and Sairy -- The Holler at Night -- Shack Talk -- Trials -- Mrs. Trepid -- Decisions -- Nightmares -- Medicine -- Paddling and Hiking -- Z's Report -- Bearings -- Stiff -- A Long Chain -- Word Pictures -- Surveying -- The Worrywarts -- Babies in the Box -- Shopping -- Dorkhead -- Loops -- Progress -- The Rock -- Stones in the Holler -- Running -- More Shopping -- Underwater -- The Feeling -- Z -- The One-Log Raft -- The Dunces -- Slow Motion -- On the Road -- On the River -- The Soggy Heart -- Preparations -- Investments -- Hospital Talk -- Mr. Trepid's Adventure -- Jewels -- Mission-Accomplished Cake -- Appraisals -- Conversations in the Night -- Dreams.
\{http://purl.org/dc/elements/1.1/\}format 310 p. ; 22 cm.
\{http://purl.org/dc/elements/1.1/\}identifier 0060277327
\{http://purl.org/dc/elements/1.1/\}identifier 9780060277321
\{http://purl.org/dc/elements/1.1/\}identifier 0060277335 (lib. bdg)
\{http://purl.org/dc/elements/1.1/\}identifier 9780060277338 (lib. bdg)
\{http://purl.org/dc/elements/1.1/\}language eng
\{http://purl.org/dc/elements/1.1/\}publisher Joanna Cotler Books/Harper Collins Publishers
\{http://purl.org/dc/elements/1.1/\}relation http://www.loc.gov/catdir/description/hc041/00066371.html
\{http://purl.org/dc/elements/1.1/\}subject Orphans--Juvenile fiction.
\{http://purl.org/dc/elements/1.1/\}subject Twins--Juvenile fiction.
\{http://purl.org/dc/elements/1.1/\}subject Brothers and sisters--Juvenile fiction.
\{http://purl.org/dc/elements/1.1/\}subject Wood-carvers--Juvenile fiction.
\{http://purl.org/dc/elements/1.1/\}subject Country life--Juvenile fiction.
\{http://purl.org/dc/elements/1.1/\}subject Voyages and travels--Juvenile fiction.
\{http://purl.org/dc/elements/1.1/\}title Ruby Holler 
\{http://purl.org/dc/elements/1.1/\}type Text
\{http://purl.org/oclc/terms/\}recordCreationDate 001109
\{http://purl.org/oclc/terms/\}recordIdentifier    00066371 
\{http://purl.org/oclc/terms/\}recordIdentifier 45487314
\{http://purl.org/dc/elements/1.1/\}creator Bridges, Ruby.
\{http://purl.org/dc/elements/1.1/\}date 1999
\{http://purl.org/dc/elements/1.1/\}description 1st ed.
\{http://purl.org/dc/elements/1.1/\}description Ruby Bridges recounts the story of her involvement, as a six-year-old, in the integration of her school in New Orleans in 1960.
\{http://purl.org/dc/elements/1.1/\}description green dot
\{http://purl.org/dc/elements/1.1/\}format 63 p. : ill. ; 28 cm.
\{http://purl.org/dc/elements/1.1/\}identifier 0590189239 (hc)
\{http://purl.org/dc/elements/1.1/\}identifier 9780590189231 (hc)
\{http://purl.org/dc/elements/1.1/\}language eng
\{http://purl.org/dc/elements/1.1/\}publisher Scholastic Press
\{http://purl.org/dc/elements/1.1/\}subject Bridges, Ruby--Juvenile literature.
\{http://purl.org/dc/elements/1.1/\}subject African American children--Louisiana--New Orleans--Biography--Juvenile literature.
\{http://purl.org/dc/elements/1.1/\}subject African Americans--Louisiana--New Orleans--Biography--Juvenile literature.
\{http://purl.org/dc/elements/1.1/\}subject School integration--Louisiana--New Orleans--Juvenile literature.
\{http://purl.org/dc/elements/1.1/\}subject New Orleans (La.)--Race relations--Juvenile literature.
\{http://purl.org/dc/elements/1.1/\}title Through my eyes 
\{http://purl.org/dc/elements/1.1/\}type Text
\{http://purl.org/oclc/terms/\}recordCreationDate 981216
\{http://purl.org/oclc/terms/\}recordIdentifier    98049242 
\{http://purl.org/oclc/terms/\}recordIdentifier 40588556
\{http://purl.org/dc/elements/1.1/\}creator Andrews, V. C. (Virginia C.)
\{http://purl.org/dc/elements/1.1/\}date c1994
\{http://purl.org/dc/elements/1.1/\}description growing up in the heart of Cajun country, the only family Ruby Landry has ever known are her loving ghuardian Jack, a drunken outcast who lives alone in a shack in the swamp.
\{http://purl.org/dc/elements/1.1/\}format 442 p. ; 23 cm.
\{http://purl.org/dc/elements/1.1/\}identifier 0671759353
\{http://purl.org/dc/elements/1.1/\}identifier 9780671759353
\{http://purl.org/dc/elements/1.1/\}language eng
\{http://purl.org/dc/elements/1.1/\}publisher Pocket Books
\{http://purl.org/dc/elements/1.1/\}subject Aileler--Roman.
\{http://purl.org/dc/elements/1.1/\}subject Luizana--Roman.
\{http://purl.org/dc/elements/1.1/\}subject Families--Fiction.
\{http://purl.org/dc/elements/1.1/\}subject Louisiana--Fiction.
\{http://purl.org/dc/elements/1.1/\}title Ruby 
\{http://purl.org/dc/elements/1.1/\}type Romantic suspense novels.
\{http://purl.org/dc/elements/1.1/\}type Text
\{http://purl.org/oclc/terms/\}recordCreationDate 940608
\{http://purl.org/oclc/terms/\}recordIdentifier    94168748 
\{http://purl.org/oclc/terms/\}recordIdentifier 29631416
\{http://purl.org/dc/elements/1.1/\}creator Emberley, Michael.
\{http://purl.org/dc/elements/1.1/\}date c1990
\{http://purl.org/dc/elements/1.1/\}description 1st ed.
\{http://purl.org/dc/elements/1.1/\}description While taking cheese pies to her Granny, Ruby, a small but tough-minded little mouse, forgets her mother's advice not to talk to cats. Story is a variation on Red Riding Hood.
\{http://purl.org/dc/elements/1.1/\}format [25] p. : col. ill. ; 21 cm.
\{http://purl.org/dc/elements/1.1/\}identifier 0316236438 (lib. bdg.) 
\{http://purl.org/dc/elements/1.1/\}identifier 9780316236430 (lib. bdg.)
\{http://purl.org/dc/elements/1.1/\}identifier 0316236608 (pbk.)
\{http://purl.org/dc/elements/1.1/\}identifier 9780316236607 (pbk.)
\{http://purl.org/dc/elements/1.1/\}language eng
\{http://purl.org/dc/elements/1.1/\}publisher Little, Brown
\{http://purl.org/dc/elements/1.1/\}subject Mice--Juvenile fiction.
\{http://purl.org/dc/elements/1.1/\}subject Cats--Juvenile fiction.
\{http://purl.org/dc/elements/1.1/\}title Ruby 
\{http://purl.org/dc/elements/1.1/\}type Text
\{http://purl.org/oclc/terms/\}recordCreationDate 890511
\{http://purl.org/oclc/terms/\}recordIdentifier    89012108 
\{http://purl.org/oclc/terms/\}recordIdentifier 19774282
\{http://purl.org/dc/elements/1.1/\}creator Ruby, Laura.
\{http://purl.org/dc/elements/1.1/\}date 2007
\{http://purl.org/dc/elements/1.1/\}description 1st ed.
\{http://purl.org/dc/elements/1.1/\}description Loopy -- Restoration -- Ballad of the Barbie feet -- Dear psycho -- Safekeeping -- Picture of health -- I'm not Julia Roberts -- The bunko bunny -- The dog next door -- Hug machine.
\{http://purl.org/dc/elements/1.1/\}format 251 p. : ill. ; 22 cm.
\{http://purl.org/dc/elements/1.1/\}identifier 0446578746
\{http://purl.org/dc/elements/1.1/\}identifier 9780446578745
\{http://purl.org/dc/elements/1.1/\}language eng
\{http://purl.org/dc/elements/1.1/\}publisher Warner Books
\{http://purl.org/dc/elements/1.1/\}relation http://www.loc.gov/catdir/toc/ecip069/2006006832.html
\{http://purl.org/dc/elements/1.1/\}subject Stepmothers--Fiction.
\{http://purl.org/dc/elements/1.1/\}subject Stepfamilies--Fiction.
\{http://purl.org/dc/elements/1.1/\}subject Illinois--Fiction.
\{http://purl.org/dc/elements/1.1/\}title I'm not Julia Roberts 
\{http://purl.org/dc/elements/1.1/\}title I am not Julia Roberts
\{http://purl.org/dc/elements/1.1/\}type Text
\{http://purl.org/oclc/terms/\}recordCreationDate 060227
\{http://purl.org/oclc/terms/\}recordIdentifier   2006006832
\{http://purl.org/oclc/terms/\}recordIdentifier 64442891
\{http://purl.org/dc/elements/1.1/\}creator Ruby, Laura.
\{http://purl.org/dc/elements/1.1/\}date c2007
\{http://purl.org/dc/elements/1.1/\}description 1st ed.
\{http://purl.org/dc/elements/1.1/\}description Thirteen-year-old Georgie and Bug, a year older, have been pulled apart by the demands of their newfound fame and fortune, but join forces again when a punk, vampires, a giant sloth, and other creatures come after them on the streets of a New York City of the future.
\{http://purl.org/dc/elements/1.1/\}format 325 p. ; 22 cm.
\{http://purl.org/dc/elements/1.1/\}identifier 9780060752583 (trade bdg.)
\{http://purl.org/dc/elements/1.1/\}identifier 9780060752590 (lib. bdg.)
\{http://purl.org/dc/elements/1.1/\}identifier 0060752580
\{http://purl.org/dc/elements/1.1/\}identifier 0060752599
\{http://purl.org/dc/elements/1.1/\}language eng
\{http://purl.org/dc/elements/1.1/\}publisher Eos
\{http://purl.org/dc/elements/1.1/\}relation http://www.loc.gov/catdir/enhancements/fy0911/2007008621-b.html
\{http://purl.org/dc/elements/1.1/\}relation http://www.loc.gov/catdir/enhancements/fy0911/2007008621-d.html
\{http://purl.org/dc/elements/1.1/\}subject Fame--Juvenile fiction.
\{http://purl.org/dc/elements/1.1/\}subject Wealth--Juvenile fiction.
\{http://purl.org/dc/elements/1.1/\}subject Supernatural--Juvenile fiction.
\{http://purl.org/dc/elements/1.1/\}subject Books and reading--Juvenile fiction.
\{http://purl.org/dc/elements/1.1/\}subject Cats--Juvenile fiction.
\{http://purl.org/dc/elements/1.1/\}subject Flight--Juvenile fiction.
\{http://purl.org/dc/elements/1.1/\}subject New York (N.Y.)--Juvenile fiction.
\{http://purl.org/dc/elements/1.1/\}title The chaos king 
\{http://purl.org/dc/elements/1.1/\}type Fantasy fiction.
\{http://purl.org/dc/elements/1.1/\}type Children's stories.
\{http://purl.org/dc/elements/1.1/\}type Text
\{http://purl.org/oclc/terms/\}recordCreationDate 070329
\{http://purl.org/oclc/terms/\}recordIdentifier   2007008621
\{http://purl.org/oclc/terms/\}recordIdentifier 122309000
\{http://purl.org/dc/elements/1.1/\}creator Ruby, Laura.
\{http://purl.org/dc/elements/1.1/\}date c2009
\{http://purl.org/dc/elements/1.1/\}description 1st ed.
\{http://purl.org/dc/elements/1.1/\}description Tola Riley, a high school junior, struggles to tell the truth when she and her art teacher are accused of having an affair.
\{http://purl.org/dc/elements/1.1/\}format 247 p. ; 22 cm.
\{http://purl.org/dc/elements/1.1/\}identifier 9780061243301 (trade bdg.)
\{http://purl.org/dc/elements/1.1/\}identifier 0061243302 (trade bdg.)
\{http://purl.org/dc/elements/1.1/\}identifier 9780061243325 (lib. bdg.)
\{http://purl.org/dc/elements/1.1/\}identifier 0061243329 (lib. bdg.)
\{http://purl.org/dc/elements/1.1/\}language eng
\{http://purl.org/dc/elements/1.1/\}publisher HarperTeen
\{http://purl.org/dc/elements/1.1/\}subject Teacher-student relationships--Fiction.
\{http://purl.org/dc/elements/1.1/\}subject Cyberbullying--Fiction.
\{http://purl.org/dc/elements/1.1/\}subject High schools--Fiction.
\{http://purl.org/dc/elements/1.1/\}subject Schools--Fiction.
\{http://purl.org/dc/elements/1.1/\}subject Family problems--Fiction.
\{http://purl.org/dc/elements/1.1/\}subject Divorce--Fiction.
\{http://purl.org/dc/elements/1.1/\}title Bad apple 
\{http://purl.org/dc/elements/1.1/\}type Text
\{http://purl.org/oclc/terms/\}recordCreationDate 090128
\{http://purl.org/oclc/terms/\}recordIdentifier   2009001409
\{http://purl.org/oclc/terms/\}recordIdentifier 300462588
\{http://purl.org/dc/elements/1.1/\}creator Ruby, Laura.
\{http://purl.org/dc/elements/1.1/\}date c2006
\{http://purl.org/dc/elements/1.1/\}description 1st ed.
\{http://purl.org/dc/elements/1.1/\}description In a future New York where most people can fly and cats are a rarity, a nondescript resident of Hope House for the Homeless and Hopeless discovers that although she is shunned as a "leadfoot," she has the surprising ability to become invisible.
\{http://purl.org/dc/elements/1.1/\}format 327 p. ; 22 cm.
\{http://purl.org/dc/elements/1.1/\}identifier 0060752556
\{http://purl.org/dc/elements/1.1/\}identifier 0060752564 (lib. bdg.)
\{http://purl.org/dc/elements/1.1/\}identifier 9780060752552
\{http://purl.org/dc/elements/1.1/\}identifier 9780060752569
\{http://purl.org/dc/elements/1.1/\}language eng
\{http://purl.org/dc/elements/1.1/\}publisher Eos
\{http://purl.org/dc/elements/1.1/\}relation http://www.loc.gov/catdir/toc/ecip0518/2005023170.html
\{http://purl.org/dc/elements/1.1/\}relation http://www.loc.gov/catdir/enhancements/fy0910/2005023170-b.html
\{http://purl.org/dc/elements/1.1/\}relation http://www.loc.gov/catdir/enhancements/fy0910/2005023170-d.html
\{http://purl.org/dc/elements/1.1/\}subject Orphans--Juvenile fiction.
\{http://purl.org/dc/elements/1.1/\}subject Orphanages--Juvenile fiction.
\{http://purl.org/dc/elements/1.1/\}subject Cats--Juvenile fiction.
\{http://purl.org/dc/elements/1.1/\}subject Flight--Juvenile fiction.
\{http://purl.org/dc/elements/1.1/\}subject Gangsters--Juvenile fiction.
\{http://purl.org/dc/elements/1.1/\}subject New York (N.Y.)--Juvenile fiction.
\{http://purl.org/dc/elements/1.1/\}title The Wall and the Wing 
\{http://purl.org/dc/elements/1.1/\}type Text
\{http://purl.org/oclc/terms/\}recordCreationDate 050816
\{http://purl.org/oclc/terms/\}recordIdentifier   2005023170
\{http://purl.org/oclc/terms/\}recordIdentifier 61451492
\{http://purl.org/dc/elements/1.1/\}creator Ruby, Laura.
\{http://purl.org/dc/elements/1.1/\}date c2006
\{http://purl.org/dc/elements/1.1/\}description 1st ed.
\{http://purl.org/dc/elements/1.1/\}description Sixteen-year-old high school senior Audrey is humiliated when a compromising photograph of her is sent around her school, but she discovers a toughness within her that she never knew she had.
\{http://purl.org/dc/elements/1.1/\}format 274 p. ; 22 cm.
\{http://purl.org/dc/elements/1.1/\}identifier 0060882239 (trade bdg.)
\{http://purl.org/dc/elements/1.1/\}identifier 9780060882235 (trade bdg.)
\{http://purl.org/dc/elements/1.1/\}identifier 0060882247 (lib. bdg.)
\{http://purl.org/dc/elements/1.1/\}identifier 9780060882242 (lib. bdg.)
\{http://purl.org/dc/elements/1.1/\}language eng
\{http://purl.org/dc/elements/1.1/\}publisher HarperTempest
\{http://purl.org/dc/elements/1.1/\}subject Interpersonal relations--Juvenile fiction.
\{http://purl.org/dc/elements/1.1/\}subject Sex--Juvenile fiction.
\{http://purl.org/dc/elements/1.1/\}subject Conduct of life--Juvenile fiction.
\{http://purl.org/dc/elements/1.1/\}subject High schools--Juvenile fiction.
\{http://purl.org/dc/elements/1.1/\}subject Schools--Juvenile fiction.
\{http://purl.org/dc/elements/1.1/\}title Good girls 
\{http://purl.org/dc/elements/1.1/\}type Text
\{http://purl.org/oclc/terms/\}recordCreationDate 060125
\{http://purl.org/oclc/terms/\}recordIdentifier   2006000340
\{http://purl.org/oclc/terms/\}recordIdentifier 63187390

    \end{Verbatim}

    \begin{Verbatim}[commandchars=\\\{\}]
{\color{incolor}In [{\color{incolor}63}]:} \PY{n}{table} \PY{o}{=} \PY{n}{tree}\PY{o}{.}\PY{n}{findall}\PY{p}{(}\PY{l+s+s1}{\PYZsq{}}\PY{l+s+s1}{.//}\PY{l+s+s1}{\PYZob{}}\PY{l+s+s1}{http://purl.org/dc/elements/1.1/\PYZcb{}identifier}\PY{l+s+s1}{\PYZsq{}}\PY{p}{)}
         \PY{k}{for} \PY{n}{child} \PY{o+ow}{in} \PY{n}{table}\PY{p}{:}
             \PY{n+nb}{print}\PY{p}{(}\PY{n}{child}\PY{o}{.}\PY{n}{tag}\PY{p}{,} \PY{n}{child}\PY{o}{.}\PY{n}{text}\PY{p}{)}
\end{Verbatim}


    \begin{Verbatim}[commandchars=\\\{\}]
\{http://purl.org/dc/elements/1.1/\}identifier 0764290762
\{http://purl.org/dc/elements/1.1/\}identifier 9780764290763
\{http://purl.org/dc/elements/1.1/\}identifier 0764222228
\{http://purl.org/dc/elements/1.1/\}identifier 9780764222221
\{http://purl.org/dc/elements/1.1/\}identifier 0060277327
\{http://purl.org/dc/elements/1.1/\}identifier 9780060277321
\{http://purl.org/dc/elements/1.1/\}identifier 0060277335 (lib. bdg)
\{http://purl.org/dc/elements/1.1/\}identifier 9780060277338 (lib. bdg)
\{http://purl.org/dc/elements/1.1/\}identifier 0590189239 (hc)
\{http://purl.org/dc/elements/1.1/\}identifier 9780590189231 (hc)
\{http://purl.org/dc/elements/1.1/\}identifier 0671759353
\{http://purl.org/dc/elements/1.1/\}identifier 9780671759353
\{http://purl.org/dc/elements/1.1/\}identifier 0316236438 (lib. bdg.) 
\{http://purl.org/dc/elements/1.1/\}identifier 9780316236430 (lib. bdg.)
\{http://purl.org/dc/elements/1.1/\}identifier 0316236608 (pbk.)
\{http://purl.org/dc/elements/1.1/\}identifier 9780316236607 (pbk.)
\{http://purl.org/dc/elements/1.1/\}identifier 0446578746
\{http://purl.org/dc/elements/1.1/\}identifier 9780446578745
\{http://purl.org/dc/elements/1.1/\}identifier 9780060752583 (trade bdg.)
\{http://purl.org/dc/elements/1.1/\}identifier 9780060752590 (lib. bdg.)
\{http://purl.org/dc/elements/1.1/\}identifier 0060752580
\{http://purl.org/dc/elements/1.1/\}identifier 0060752599
\{http://purl.org/dc/elements/1.1/\}identifier 9780061243301 (trade bdg.)
\{http://purl.org/dc/elements/1.1/\}identifier 0061243302 (trade bdg.)
\{http://purl.org/dc/elements/1.1/\}identifier 9780061243325 (lib. bdg.)
\{http://purl.org/dc/elements/1.1/\}identifier 0061243329 (lib. bdg.)
\{http://purl.org/dc/elements/1.1/\}identifier 0060752556
\{http://purl.org/dc/elements/1.1/\}identifier 0060752564 (lib. bdg.)
\{http://purl.org/dc/elements/1.1/\}identifier 9780060752552
\{http://purl.org/dc/elements/1.1/\}identifier 9780060752569
\{http://purl.org/dc/elements/1.1/\}identifier 0060882239 (trade bdg.)
\{http://purl.org/dc/elements/1.1/\}identifier 9780060882235 (trade bdg.)
\{http://purl.org/dc/elements/1.1/\}identifier 0060882247 (lib. bdg.)
\{http://purl.org/dc/elements/1.1/\}identifier 9780060882242 (lib. bdg.)

    \end{Verbatim}

    \begin{Verbatim}[commandchars=\\\{\}]
{\color{incolor}In [{\color{incolor}64}]:} \PY{n}{relation} \PY{o}{=} \PY{n}{tree}\PY{o}{.}\PY{n}{findall}\PY{p}{(}\PY{l+s+s1}{\PYZsq{}}\PY{l+s+s1}{.//}\PY{l+s+s1}{\PYZob{}}\PY{l+s+s1}{http://purl.org/dc/elements/1.1/\PYZcb{}identifier}\PY{l+s+s1}{\PYZsq{}}\PY{p}{)}
         \PY{k}{for} \PY{n}{elem} \PY{o+ow}{in} \PY{n}{relation}\PY{p}{:}
             \PY{n+nb}{print}\PY{p}{(}\PY{n}{elem}\PY{o}{.}\PY{n}{tag}\PY{p}{,} \PY{n}{elem}\PY{o}{.}\PY{n}{text}\PY{p}{)}
\end{Verbatim}


    \begin{Verbatim}[commandchars=\\\{\}]
\{http://purl.org/dc/elements/1.1/\}identifier 0764290762
\{http://purl.org/dc/elements/1.1/\}identifier 9780764290763
\{http://purl.org/dc/elements/1.1/\}identifier 0764222228
\{http://purl.org/dc/elements/1.1/\}identifier 9780764222221
\{http://purl.org/dc/elements/1.1/\}identifier 0060277327
\{http://purl.org/dc/elements/1.1/\}identifier 9780060277321
\{http://purl.org/dc/elements/1.1/\}identifier 0060277335 (lib. bdg)
\{http://purl.org/dc/elements/1.1/\}identifier 9780060277338 (lib. bdg)
\{http://purl.org/dc/elements/1.1/\}identifier 0590189239 (hc)
\{http://purl.org/dc/elements/1.1/\}identifier 9780590189231 (hc)
\{http://purl.org/dc/elements/1.1/\}identifier 0671759353
\{http://purl.org/dc/elements/1.1/\}identifier 9780671759353
\{http://purl.org/dc/elements/1.1/\}identifier 0316236438 (lib. bdg.) 
\{http://purl.org/dc/elements/1.1/\}identifier 9780316236430 (lib. bdg.)
\{http://purl.org/dc/elements/1.1/\}identifier 0316236608 (pbk.)
\{http://purl.org/dc/elements/1.1/\}identifier 9780316236607 (pbk.)
\{http://purl.org/dc/elements/1.1/\}identifier 0446578746
\{http://purl.org/dc/elements/1.1/\}identifier 9780446578745
\{http://purl.org/dc/elements/1.1/\}identifier 9780060752583 (trade bdg.)
\{http://purl.org/dc/elements/1.1/\}identifier 9780060752590 (lib. bdg.)
\{http://purl.org/dc/elements/1.1/\}identifier 0060752580
\{http://purl.org/dc/elements/1.1/\}identifier 0060752599
\{http://purl.org/dc/elements/1.1/\}identifier 9780061243301 (trade bdg.)
\{http://purl.org/dc/elements/1.1/\}identifier 0061243302 (trade bdg.)
\{http://purl.org/dc/elements/1.1/\}identifier 9780061243325 (lib. bdg.)
\{http://purl.org/dc/elements/1.1/\}identifier 0061243329 (lib. bdg.)
\{http://purl.org/dc/elements/1.1/\}identifier 0060752556
\{http://purl.org/dc/elements/1.1/\}identifier 0060752564 (lib. bdg.)
\{http://purl.org/dc/elements/1.1/\}identifier 9780060752552
\{http://purl.org/dc/elements/1.1/\}identifier 9780060752569
\{http://purl.org/dc/elements/1.1/\}identifier 0060882239 (trade bdg.)
\{http://purl.org/dc/elements/1.1/\}identifier 9780060882235 (trade bdg.)
\{http://purl.org/dc/elements/1.1/\}identifier 0060882247 (lib. bdg.)
\{http://purl.org/dc/elements/1.1/\}identifier 9780060882242 (lib. bdg.)

    \end{Verbatim}

    \hypertarget{xpath}{%
\subsection{XPATH}\label{xpath}}

    XPath es un lenguaje que permite construir expresiones que recorren y
procesan un documento XML. La idea es parecida a las expresiones
regulares para seleccionar partes de un texto sin atributos. XPath
permite buscar y seleccionar teniendo en cuenta la estructura jerárquica
del XML

    Syntax

Meaning

{tag}

Selects all child elements with the given tag. For example, {spam}
selects all child elements named {spam}, and {spam/egg} selects all
grandchildren named {egg} in all children named {spam}.

*

Selects all child elements. For example, */egg selects all grandchildren
named {egg}.

{.}

Selects the current node. This is mostly useful at the beginning of the
path, to indicate that it's a relative path.

{//}

Selects all subelements, on all levels beneath the current element. For
example, {.//egg} selects all {egg} elements in the entire tree.

{..}

Selects the parent element.

{{[}@attrib{]}}

Selects all elements that have the given attribute.

{{[}@attrib='value'{]}}

Selects all elements for which the given attribute has the given value.
The value cannot contain quotes.

{{[}tag{]}}

Selects all elements that have a child named {tag}. Only immediate
children are supported.

{{[}tag=`text'{]}}

Selects all elements that have a child named {tag} whose complete text
content, including descendants, equals the given {text}.

{{[}position{]}}

Selects all elements that are located at the given position. The
position can be either an integer (1 is the first position), the
expression {last()} (for the last position), or a position relative to
the last position (e.g. {last()-1}).

    Como ves, hay que ir entendiendo la jerarquía del XML para poder obtener
la información.

¿Puedes obtener los títulos de los recursos descritos en el XML?

    \textbf{Ayuda}

`//' para indicar que empiece a buscar desde el elemento actual +
tipo+nombre del elemento a buscar (\{http://purl.org/dc/elements/1.1/\}
title)

    \begin{Verbatim}[commandchars=\\\{\}]
{\color{incolor}In [{\color{incolor}25}]:} \PY{n}{relation} \PY{o}{=} \PY{n}{tree}\PY{o}{.}\PY{n}{findall}\PY{p}{(}\PY{l+s+s1}{\PYZsq{}}\PY{l+s+s1}{.//}\PY{l+s+s1}{\PYZob{}}\PY{l+s+s1}{http://purl.org/dc/elements/1.1/\PYZcb{}title}\PY{l+s+s1}{\PYZsq{}}\PY{p}{)}
         \PY{k}{for} \PY{n}{elem} \PY{o+ow}{in} \PY{n}{relation}\PY{p}{:}
             \PY{n+nb}{print}\PY{p}{(}\PY{n}{elem}\PY{o}{.}\PY{n}{tag}\PY{p}{,} \PY{n}{elem}\PY{o}{.}\PY{n}{text}\PY{p}{)}
\end{Verbatim}


    \begin{Verbatim}[commandchars=\\\{\}]
\{http://purl.org/dc/elements/1.1/\}title Ruby 
\{http://purl.org/dc/elements/1.1/\}title Ruby Holler 
\{http://purl.org/dc/elements/1.1/\}title Through my eyes 
\{http://purl.org/dc/elements/1.1/\}title Ruby 
\{http://purl.org/dc/elements/1.1/\}title Ruby 
\{http://purl.org/dc/elements/1.1/\}title I'm not Julia Roberts 
\{http://purl.org/dc/elements/1.1/\}title I am not Julia Roberts
\{http://purl.org/dc/elements/1.1/\}title The chaos king 
\{http://purl.org/dc/elements/1.1/\}title Bad apple 
\{http://purl.org/dc/elements/1.1/\}title The Wall and the Wing 
\{http://purl.org/dc/elements/1.1/\}title Good girls 

    \end{Verbatim}

    Haz lo mismo utilizando namespace

    \begin{Verbatim}[commandchars=\\\{\}]
{\color{incolor}In [{\color{incolor}65}]:} \PY{n}{namespaces} \PY{o}{=} \PY{p}{\PYZob{}}\PY{l+s+s1}{\PYZsq{}}\PY{l+s+s1}{dc}\PY{l+s+s1}{\PYZsq{}}\PY{p}{:} \PY{l+s+s1}{\PYZsq{}}\PY{l+s+s1}{http://purl.org/dc/elements/1.1/}\PY{l+s+s1}{\PYZsq{}}\PY{p}{\PYZcb{}} \PY{c+c1}{\PYZsh{} add more as needed}
         \PY{n}{relation} \PY{o}{=} \PY{n}{tree}\PY{o}{.}\PY{n}{findall}\PY{p}{(}\PY{l+s+s1}{\PYZsq{}}\PY{l+s+s1}{.//dc:title}\PY{l+s+s1}{\PYZsq{}}\PY{p}{,}\PY{n}{namespaces}\PY{p}{)}
         \PY{k}{for} \PY{n}{elem} \PY{o+ow}{in} \PY{n}{relation}\PY{p}{:}
             \PY{n+nb}{print}\PY{p}{(}\PY{n}{elem}\PY{o}{.}\PY{n}{tag}\PY{p}{,} \PY{n}{elem}\PY{o}{.}\PY{n}{text}\PY{p}{)}
\end{Verbatim}


    \begin{Verbatim}[commandchars=\\\{\}]
\{http://purl.org/dc/elements/1.1/\}title Ruby 
\{http://purl.org/dc/elements/1.1/\}title Ruby Holler 
\{http://purl.org/dc/elements/1.1/\}title Through my eyes 
\{http://purl.org/dc/elements/1.1/\}title Ruby 
\{http://purl.org/dc/elements/1.1/\}title Ruby 
\{http://purl.org/dc/elements/1.1/\}title I'm not Julia Roberts 
\{http://purl.org/dc/elements/1.1/\}title I am not Julia Roberts
\{http://purl.org/dc/elements/1.1/\}title The chaos king 
\{http://purl.org/dc/elements/1.1/\}title Bad apple 
\{http://purl.org/dc/elements/1.1/\}title The Wall and the Wing 
\{http://purl.org/dc/elements/1.1/\}title Good girls 

    \end{Verbatim}

    Ejemplo con EML

    \begin{Verbatim}[commandchars=\\\{\}]
{\color{incolor}In [{\color{incolor}27}]:} \PY{k+kn}{import} \PY{n+nn}{requests}
         
         \PY{n}{response} \PY{o}{=} \PY{n}{requests}\PY{o}{.}\PY{n}{get}\PY{p}{(}\PY{l+s+s1}{\PYZsq{}}\PY{l+s+s1}{https://zenodo.org/record/841691/files/amt\PYZus{}prototype.xml}\PY{l+s+s1}{\PYZsq{}}\PY{p}{)}
         \PY{k}{if} \PY{n}{response}\PY{o}{.}\PY{n}{status\PYZus{}code} \PY{o}{==} \PY{l+m+mi}{200}\PY{p}{:}
             \PY{k}{with} \PY{n+nb}{open}\PY{p}{(}\PY{l+s+s2}{\PYZdq{}}\PY{l+s+s2}{./amt\PYZus{}prototype.xml}\PY{l+s+s2}{\PYZdq{}}\PY{p}{,} \PY{l+s+s1}{\PYZsq{}}\PY{l+s+s1}{wb}\PY{l+s+s1}{\PYZsq{}}\PY{p}{)} \PY{k}{as} \PY{n}{f}\PY{p}{:}
                 \PY{n}{f}\PY{o}{.}\PY{n}{write}\PY{p}{(}\PY{n}{response}\PY{o}{.}\PY{n}{content}\PY{p}{)}
                 
\end{Verbatim}


    \begin{Verbatim}[commandchars=\\\{\}]
{\color{incolor}In [{\color{incolor}28}]:} \PY{n}{ls}
\end{Verbatim}


    \begin{Verbatim}[commandchars=\\\{\}]
amt\_prototype.xml  dublincore-example.xml  metadataIntro.ipynb

    \end{Verbatim}

    En estándares más complejos, el xml de base puede tener una jerarquía
anidada, como es el caso de EML. Entonces, cada elemento puede tener de
0 a N ``hijos'', formando nuevos árboles.

    \begin{Verbatim}[commandchars=\\\{\}]
{\color{incolor}In [{\color{incolor}4}]:} \PY{n}{tree} \PY{o}{=} \PY{n}{ET}\PY{o}{.}\PY{n}{parse}\PY{p}{(}\PY{l+s+s1}{\PYZsq{}}\PY{l+s+s1}{amt\PYZus{}prototype.xml}\PY{l+s+s1}{\PYZsq{}}\PY{p}{)}
        \PY{n}{root} \PY{o}{=} \PY{n}{tree}\PY{o}{.}\PY{n}{getroot}\PY{p}{(}\PY{p}{)}
        
        \PY{k}{for} \PY{n}{table} \PY{o+ow}{in} \PY{n}{root}\PY{o}{.}\PY{n}{getiterator}\PY{p}{(}\PY{p}{)}\PY{p}{:}
            \PY{k}{for} \PY{n}{child} \PY{o+ow}{in} \PY{n}{table}\PY{p}{:}
                \PY{k}{if} \PY{n+nb}{len}\PY{p}{(}\PY{n}{child}\PY{p}{)}\PY{o}{==}\PY{l+m+mi}{0}\PY{p}{:}
                    \PY{n+nb}{print}\PY{p}{(}\PY{n}{child}\PY{o}{.}\PY{n}{tag}\PY{p}{,} \PY{l+s+s2}{\PYZdq{}}\PY{l+s+s2}{|}\PY{l+s+s2}{\PYZdq{}}\PY{p}{,} \PY{n}{child}\PY{o}{.}\PY{n}{text}\PY{p}{)}
\end{Verbatim}


    \begin{Verbatim}[commandchars=\\\{\}]
alternateIdentifier | 
10.5281/zenodo.841183

title | water reservoir of Cuerda del Pozo
organizationName | IFCA
electronicMailAddress | marco@ifca.unican.es
salutation | Mr
givenName | Jesus Marco
surName | De Lucas
deliveryPoint | Avda Castros s/n
city | Santander
postalCode | 39005
country | Spain
organizationName | IFCA
electronicMailAddress | aguilarf@ifca.unican.es
role | guardian
givenName | Fernando
surName | Aguilar
deliveryPoint | Avda Castros s/n
city | Santander
postalCode | 39005
country | Spain
para | The CTD 60 is a precision probe for oceanographic and limnological measurements of physical, chemical and optical parameters up to a depth of 2000 m. It allows the simultaneous measurement of following parameters: Pressure (depth), temperature, conductivity, raw O2, REDOX, dissolved oxygen, pH, Oxigen Saturation, Salinity.
keyword | measure
keyword | water reservoir
keyword | sensor
keyword | physical and chemical parameters
geographicDescription | water reservoir
westBoundingCoordinate | -3.75
eastBoundingCoordinate | -2.375
northBoundingCoordinate | 42.0
southBoundingCoordinate | 40.875
calendarDate | 2010
calendarDate | 2010
organizationName | IFCA
electronicMailAddress | garciad@ifca.unican.es
givenName | Daniel
surName | Garcia
deliveryPoint | Avda Castros S/N
city | Santander
postalCode | 39005
country | Spain
instrumentation | CTD60M-Probe. It allows the simultaneous measurement of following parameters: Pressure (depth), temperature, conductivity, raw O2, REDOX, dissolved oxygen, pH, Oxigen Saturation, Salinity.
title | stationary measures
para | measures taken at a stationary depth of about 4 meters with a CTD 60. Measurements of physical, chemical and optical parameter.
entityName | AMT
entityDescription | Measurement of following parameters: Pressure (depth), temperature, conductivity, raw O2, REDOX, dissolved oxygen, pH, Oxigen Saturation, Salinity.
objectName | amt.csv
size | 4275737
authentication | f38ded28383d9f69af7cb9c98aed798b
encodingMethod | ASCII text
characterEncoding | us-ascii
numHeaderLines | 1
numFooterLines | 0
attributeOrientation | column
recordDelimiter | \textbackslash{}n
physicalLineDelimiter | \textbackslash{}n
fieldDelimiter | ;
quoteCharacter | "
url | http://doriiie02.ifca.es/datasets/amt.csv
title | ROEM +
personnel | Agustin Monteoliva
abstract | Life + Project{\ldots} eutrophication
funding | EC
descriptor | Ecology
attributeName | date
attributeLabel | date
attributeDefinition | date of measure
storageType | string
formatString | YYYY-MM-DD , hh:mm
dateTimePrecision | 1
minimum | 2010-01-01 , 00:00
maximum | 2016-12-31 , 23:59
attributeName | Temperature
attributeLabel | Temp
attributeDefinition | Temperature
storageType | double
precision | 0.01
standardUnit | celsius
numberType | real
minimum | -2.0
maximum | 32.0
para | This parameter is filtered from raw when value is less than -1.0 or higher than 35.0
references | 1827473819284
para | Temperature calibration
description | Calibration step for temperature
attributeName | Press
attributeLabel | Press
attributeDefinition | Press (Depth)
storageType | double
precision | 0.1
customUnit | dbar
numberType | real
minimum | 0
maximum | 50
para | Parameter is filtered when Press * 1.02 less than 0.0 or Press * 1.02 is higher than  30.0
references | 1827473819284
attributeName | Conductivity
attributeLabel | Cond
attributeDefinition | Conductivity of an electrolyte solution is a measure of its ability to conduct electricity. The SI unit of conductivity is siemens per meter (S/m).
storageType | double
precision | 0.005
customUnit | mS/cm
numberType | real
minimum | 0
maximum | 6
para | Parameter is filtered when is less than 0.0001 or higher than 0.3
references | 1827473819284
attributeName | Salinity
attributeLabel | Salinity
attributeDefinition | Salinity of water cannot be measured directly by one single sensor, it is calculated from the individual sensors for pressure, temperature and conductivity
storageType | Double
precision | 0.0001
customUnit | psu
numberType | real
minimum | 0
maximum | 0.1
references | 1827473819284
attributeName | Dissolved oxygen
attributeLabel | DO
attributeDefinition | Dissolved oxygen refers to the level of free
storageType | double
precision | 0.1
standardUnit | milligramsPerLiter
numberType | real
minimum | 0
maximum | 20
para | Parameter is filtered when is less than 0 and higher than 30
references | 1827473819284
attributeName | rawO2
attributeLabel | rawO2
attributeDefinition | dissolved oxygen
storageType | double
precision | 10
standardUnit | dimensionless
numberType | real
minimum | 0
maximum | 5000
references | 1827473819284
attributeName | Oxygen Saturation
attributeLabel | OxySat
attributeDefinition | relative measure of the amount of oxygen that is dissolved or carried in a given medium. \%
storageType | double
precision | 1
standardUnit | dimensionless
numberType | real
minimum | 0
maximum | 120
para | Parameter is filtered when is less than 0  or higher than 200
references | 1827473819284
attributeName | ph
attributeLabel | ph
attributeDefinition | numeric scale used to specify the acidity or basicity
storageType | double
precision | 0.1
standardUnit | dimensionless
numberType | real
minimum | 4
maximum | 10
para | Parameter is filtered when is less tha 5 or higher than10
references | 1827473819284
attributeName | Redox
attributeLabel | redox
attributeDefinition | Redox
storageType | double
precision | 10
standardUnit | dimensionless
numberType | real
minimum | 0
maximum | 5000
references | 1827473819284
title | Peaks Detector
license | GPL
version | 1.0
para | Detects and filter peaks from parameters defined in a CSV file 
references | 1465302104227
programmingLanguage | Python 2.7.3
mediumName | doriiie02 Server HDD
description | decibar, Units derived from the bar.
description | Conductivity of an electrolyte solution is a measure of its ability to conduct electricity. The SI unit of conductivity is siemens per meter (S/m).
description | practical salinity unit

    \end{Verbatim}

    Explora un poco: Nombre del proyecto, autores, lista de
atributos\ldots{}

    \begin{Verbatim}[commandchars=\\\{\}]
{\color{incolor}In [{\color{incolor}31}]:} \PY{c+c1}{\PYZsh{}elementos = tree.findall(\PYZsq{}/dataset[1]/creator/individualName/salutation\PYZsq{})}
         \PY{n}{elementos} \PY{o}{=} \PY{n}{tree}\PY{o}{.}\PY{n}{findall}\PY{p}{(}\PY{l+s+s1}{\PYZsq{}}\PY{l+s+s1}{.//attributeList/attribute[@id=}\PY{l+s+s1}{\PYZdq{}}\PY{l+s+s1}{1465311292527}\PY{l+s+s1}{\PYZdq{}}\PY{l+s+s1}{]/attributeName}\PY{l+s+s1}{\PYZsq{}}\PY{p}{)}
         \PY{k}{for} \PY{n}{e} \PY{o+ow}{in} \PY{n}{elementos}\PY{p}{:}
             \PY{n+nb}{print}\PY{p}{(}\PY{n}{e}\PY{o}{.}\PY{n}{tag} \PY{o}{+} \PY{l+s+s2}{\PYZdq{}}\PY{l+s+s2}{:}\PY{l+s+s2}{\PYZdq{}}\PY{p}{,} \PY{n}{e}\PY{o}{.}\PY{n}{text}\PY{p}{)}
\end{Verbatim}


    \begin{Verbatim}[commandchars=\\\{\}]
attributeName: date

    \end{Verbatim}

    \begin{Verbatim}[commandchars=\\\{\}]
{\color{incolor}In [{\color{incolor}32}]:} \PY{n}{elementos} \PY{o}{=} \PY{n}{tree}\PY{o}{.}\PY{n}{findall}\PY{p}{(}\PY{l+s+s1}{\PYZsq{}}\PY{l+s+s1}{.//dataset}\PY{l+s+s1}{\PYZsq{}}\PY{p}{)}
         \PY{k}{for} \PY{n}{e} \PY{o+ow}{in} \PY{n}{elementos}\PY{p}{:}
             \PY{n+nb}{print}\PY{p}{(}\PY{n}{e}\PY{o}{.}\PY{n}{tag} \PY{o}{+} \PY{l+s+s2}{\PYZdq{}}\PY{l+s+s2}{:}\PY{l+s+s2}{\PYZdq{}}\PY{p}{,} \PY{n}{e}\PY{o}{.}\PY{n}{text}\PY{p}{)}
             \PY{k}{for} \PY{n}{i} \PY{o+ow}{in} \PY{n}{e}\PY{o}{.}\PY{n}{getiterator}\PY{p}{(}\PY{p}{)}\PY{p}{:}
                 \PY{n+nb}{print}\PY{p}{(}\PY{n}{i}\PY{o}{.}\PY{n}{tag} \PY{o}{+} \PY{l+s+s2}{\PYZdq{}}\PY{l+s+s2}{:}\PY{l+s+s2}{\PYZdq{}}\PY{p}{,} \PY{n}{i}\PY{o}{.}\PY{n}{text}\PY{p}{)}
\end{Verbatim}


    \begin{Verbatim}[commandchars=\\\{\}]
dataset:  

dataset:  

title: water reservoir of Cuerda del Pozo
creator:  
individualName: None
salutation: Mr
givenName: Jesus Marco
surName: De Lucas
organizationName: IFCA
address: None
deliveryPoint: Avda Castros s/n
city: Santander
postalCode: 39005
country: Spain
electronicMailAddress: marco@ifca.unican.es
associatedParty: None
individualName: None
givenName: Fernando
surName: Aguilar
organizationName: IFCA
address: None
deliveryPoint: Avda Castros s/n
city: Santander
postalCode: 39005
country: Spain
electronicMailAddress: aguilarf@ifca.unican.es
role: guardian
abstract: None
para: The CTD 60 is a precision probe for oceanographic and limnological measurements of physical, chemical and optical parameters up to a depth of 2000 m. It allows the simultaneous measurement of following parameters: Pressure (depth), temperature, conductivity, raw O2, REDOX, dissolved oxygen, pH, Oxigen Saturation, Salinity.
keywordSet: None
keyword: measure
keyword: water reservoir
keyword: sensor
keyword: physical and chemical parameters
coverage: None
geographicCoverage: None
geographicDescription: water reservoir
boundingCoordinates: None
westBoundingCoordinate: -3.75
eastBoundingCoordinate: -2.375
northBoundingCoordinate: 42.0
southBoundingCoordinate: 40.875
temporalCoverage: None
rangeOfDates: None
beginDate: None
calendarDate: 2010
endDate: None
calendarDate: 2010
contact: None
individualName: None
givenName: Daniel
surName: Garcia
organizationName: IFCA
address: None
deliveryPoint: Avda Castros S/N
city: Santander
postalCode: 39005
country: Spain
electronicMailAddress: garciad@ifca.unican.es
methods: None
methodStep: None
description: None
section: None
title: stationary measures
para: measures taken at a stationary depth of about 4 meters with a CTD 60. Measurements of physical, chemical and optical parameter.
instrumentation: CTD60M-Probe. It allows the simultaneous measurement of following parameters: Pressure (depth), temperature, conductivity, raw O2, REDOX, dissolved oxygen, pH, Oxigen Saturation, Salinity.
dataTable: None
entityName: AMT
entityDescription: Measurement of following parameters: Pressure (depth), temperature, conductivity, raw O2, REDOX, dissolved oxygen, pH, Oxigen Saturation, Salinity.
physical: 

objectName: amt.csv
size: 4275737
authentication: f38ded28383d9f69af7cb9c98aed798b
encodingMethod: ASCII text
characterEncoding: us-ascii
dataFormat:  

textFormat: 

numHeaderLines: 1
numFooterLines: 0
attributeOrientation: column
simpleDelimited: 

fieldDelimiter: ;
quoteCharacter: "
recordDelimiter: \textbackslash{}n
physicalLineDelimiter: \textbackslash{}n
distribution: 

online: 

url: http://doriiie02.ifca.es/datasets/amt.csv
project: 

researchProjectType: 

title: ROEM +
personnel: Agustin Monteoliva
abstract: Life + Project{\ldots} eutrophication
funding: EC
studyAreaDescription: 

descriptor: Ecology
attributeList: 


attribute: 

attributeName: date
attributeLabel: date
attributeDefinition: date of measure
storageType: string
measurementScale: 

dateTime: 

formatString: YYYY-MM-DD , hh:mm
dateTimePrecision: 1
dateTimeDomain: 
 
bounds: 

minimum: 2010-01-01 , 00:00
maximum: 2016-12-31 , 23:59
attribute: None
attributeName: Temperature
attributeLabel: Temp
attributeDefinition: Temperature
storageType: double
measurementScale: 

ratio: 

unit: 

standardUnit: celsius
precision: 0.01
numericDomain: 

numberType: real
bounds: 

minimum: -2.0
maximum: 32.0
method: 

qualityControl: 

description: 

para: This parameter is filtered from raw when value is less than -1.0 or higher than 35.0
software: 

references: 1827473819284
methodStep: 

description: 

para: Temperature calibration
protocol: 

proceduralStep: 

description: Calibration step for temperature
attribute: None
attributeName: Press
attributeLabel: Press
attributeDefinition: Press (Depth)
storageType: double
measurementScale: None
interval: None
unit: None
customUnit: dbar
precision: 0.1
numericDomain: None
numberType: real
bounds: None
minimum: 0
maximum: 50
method: 

qualityControl: 

description: 

para: Parameter is filtered when Press * 1.02 less than 0.0 or Press * 1.02 is higher than  30.0
software: 

references: 1827473819284
attribute: None
attributeName: Conductivity
attributeLabel: Cond
attributeDefinition: Conductivity of an electrolyte solution is a measure of its ability to conduct electricity. The SI unit of conductivity is siemens per meter (S/m).
storageType: double
measurementScale: None
interval: None
unit: None
customUnit: mS/cm
precision: 0.005
numericDomain: None
numberType: real
bounds: None
minimum: 0
maximum: 6
method: 

qualityControl: 

description: 

para: Parameter is filtered when is less than 0.0001 or higher than 0.3
software: 

references: 1827473819284
attribute: None
attributeName: Salinity
attributeLabel: Salinity
attributeDefinition: Salinity of water cannot be measured directly by one single sensor, it is calculated from the individual sensors for pressure, temperature and conductivity
storageType: Double
measurementScale: None
interval: None
unit: None
customUnit: psu
precision: 0.0001
numericDomain: None
numberType: real
bounds: None
minimum: 0
maximum: 0.1
method: 

qualityControl: 

software: 

references: 1827473819284
attribute: None
attributeName: Dissolved oxygen
attributeLabel: DO
attributeDefinition: Dissolved oxygen refers to the level of free
storageType: double
measurementScale: None
interval: None
unit: None
standardUnit: milligramsPerLiter
precision: 0.1
numericDomain: None
numberType: real
bounds: None
minimum: 0
maximum: 20
method: 

qualityControl: 

description: 

para: Parameter is filtered when is less than 0 and higher than 30
software: 

references: 1827473819284
attribute: None
attributeName: rawO2
attributeLabel: rawO2
attributeDefinition: dissolved oxygen
storageType: double
measurementScale: None
interval: None
unit: None
standardUnit: dimensionless
precision: 10
numericDomain: None
numberType: real
bounds: None
minimum: 0
maximum: 5000
method: 

qualityControl: 

software: 

references: 1827473819284
attribute: None
attributeName: Oxygen Saturation
attributeLabel: OxySat
attributeDefinition: relative measure of the amount of oxygen that is dissolved or carried in a given medium. \%
storageType: double
measurementScale: None
ratio: None
unit: None
standardUnit: dimensionless
precision: 1
numericDomain: None
numberType: real
bounds: None
minimum: 0
maximum: 120
method: 

qualityControl: 

description: 

para: Parameter is filtered when is less than 0  or higher than 200
software: 

references: 1827473819284
attribute: None
attributeName: ph
attributeLabel: ph
attributeDefinition: numeric scale used to specify the acidity or basicity
storageType: double
measurementScale: None
ratio: None
unit: None
standardUnit: dimensionless
precision: 0.1
numericDomain: None
numberType: real
bounds: None
minimum: 4
maximum: 10
method: 

qualityControl: 

description: 

para: Parameter is filtered when is less tha 5 or higher than10
software: 

references: 1827473819284
attribute: None
attributeName: Redox
attributeLabel: redox
attributeDefinition: Redox
storageType: double
measurementScale: None
interval: None
unit: None
standardUnit: dimensionless
precision: 10
numericDomain: None
numberType: real
bounds: None
minimum: 0
maximum: 5000
method: 

qualityControl: 

software: 

references: 1827473819284

    \end{Verbatim}

    \begin{Verbatim}[commandchars=\\\{\}]
{\color{incolor}In [{\color{incolor}33}]:} \PY{n}{dataset} \PY{o}{=} \PY{n}{ET}\PY{o}{.}\PY{n}{SubElement}\PY{p}{(}\PY{n}{root}\PY{p}{,}\PY{l+s+s1}{\PYZsq{}}\PY{l+s+s1}{dataset}\PY{l+s+s1}{\PYZsq{}}\PY{p}{)}
         \PY{k}{for} \PY{n}{table} \PY{o+ow}{in} \PY{n}{dataset}\PY{o}{.}\PY{n}{getiterator}\PY{p}{(}\PY{p}{)}\PY{p}{:}
             \PY{n+nb}{print}\PY{p}{(}\PY{n}{child}\PY{o}{.}\PY{n}{tag}\PY{p}{,} \PY{n}{child}\PY{o}{.}\PY{n}{text}\PY{p}{)}
\end{Verbatim}


    \begin{Verbatim}[commandchars=\\\{\}]
description practical salinity unit

    \end{Verbatim}

    \hypertarget{ejercicio-personal}{%
\section{Ejercicio personal}\label{ejercicio-personal}}

    \hypertarget{ejercicio-1}{%
\subsection{Ejercicio 1}\label{ejercicio-1}}

    A partir del ejemplo completo del esquema de metadatos de DataCite,
muestra por pantalla los elementos que sean equivalentes a los
propuestos por Dublin Core (cada uno en una línea). Es posible que
tengas que combinar en uno varios campos del archivo de metadatos (Por
ejemplo, en coverage las coordenadas + el nombre).

\begin{itemize}
\tightlist
\item
  Title:
\item
  Creator:
\item
  Subject: ~
\item
  Description:
\item
  Publisher:
\item
  Contributor:
\item
  Date:
\item
  Type:
\item
  Format:
\item
  Identifier:
\item
  Source:
\item
  Language:
\item
  Relation:
\item
  Coverage: ~
\item
  Rights:
\end{itemize}

Recurso:
https://schema.datacite.org/meta/kernel-3.1/example/datacite-example-full-v3.1.xml

    \begin{Verbatim}[commandchars=\\\{\}]
{\color{incolor}In [{\color{incolor}3}]:} \PY{k+kn}{import} \PY{n+nn}{xml}\PY{n+nn}{.}\PY{n+nn}{etree}\PY{n+nn}{.}\PY{n+nn}{ElementTree} \PY{k}{as} \PY{n+nn}{ET}
        \PY{k+kn}{import} \PY{n+nn}{requests}
        
        \PY{n}{response} \PY{o}{=} \PY{n}{requests}\PY{o}{.}\PY{n}{get}\PY{p}{(}\PY{l+s+s1}{\PYZsq{}}\PY{l+s+s1}{https://schema.datacite.org/meta/kernel\PYZhy{}3.1/example/datacite\PYZhy{}example\PYZhy{}full\PYZhy{}v3.1.xml}\PY{l+s+s1}{\PYZsq{}}\PY{p}{)}
        \PY{k}{if} \PY{n}{response}\PY{o}{.}\PY{n}{status\PYZus{}code} \PY{o}{==} \PY{l+m+mi}{200}\PY{p}{:}
            \PY{k}{with} \PY{n+nb}{open}\PY{p}{(}\PY{l+s+s2}{\PYZdq{}}\PY{l+s+s2}{./datacite\PYZhy{}example\PYZhy{}full\PYZhy{}v3.1.xml}\PY{l+s+s2}{\PYZdq{}}\PY{p}{,} \PY{l+s+s1}{\PYZsq{}}\PY{l+s+s1}{wb}\PY{l+s+s1}{\PYZsq{}}\PY{p}{)} \PY{k}{as} \PY{n}{f}\PY{p}{:}
                \PY{n}{f}\PY{o}{.}\PY{n}{write}\PY{p}{(}\PY{n}{response}\PY{o}{.}\PY{n}{content}\PY{p}{)}
\end{Verbatim}


    \begin{Verbatim}[commandchars=\\\{\}]
{\color{incolor}In [{\color{incolor}4}]:} \PY{n}{ls}
\end{Verbatim}


    \begin{Verbatim}[commandchars=\\\{\}]
01-metadataIntro.ipynb  03-OAI-PMH-APIs.ipynb  datacite-example-full-v3.1.xml
02-DOI.ipynb            amt\_prototype.xml      dublincore-example.xml

    \end{Verbatim}

    \begin{Verbatim}[commandchars=\\\{\}]
{\color{incolor}In [{\color{incolor}7}]:} \PY{n}{tree} \PY{o}{=} \PY{n}{ET}\PY{o}{.}\PY{n}{parse}\PY{p}{(}\PY{l+s+s1}{\PYZsq{}}\PY{l+s+s1}{datacite\PYZhy{}example\PYZhy{}full\PYZhy{}v3.1.xml}\PY{l+s+s1}{\PYZsq{}}\PY{p}{)}
        \PY{n}{namespaces} \PY{o}{=} \PY{p}{\PYZob{}}\PY{l+s+s1}{\PYZsq{}}\PY{l+s+s1}{dc}\PY{l+s+s1}{\PYZsq{}}\PY{p}{:} \PY{l+s+s1}{\PYZsq{}}\PY{l+s+s1}{http://datacite.org/schema/kernel\PYZhy{}3}\PY{l+s+s1}{\PYZsq{}}\PY{p}{\PYZcb{}}
        
        \PY{n}{find} \PY{o}{=} \PY{n}{tree}\PY{o}{.}\PY{n}{findall}\PY{p}{(}\PY{l+s+s1}{\PYZsq{}}\PY{l+s+s1}{.//dc:title}\PY{l+s+s1}{\PYZsq{}}\PY{p}{,}\PY{n}{namespaces}\PY{p}{)}
        
        \PY{k}{for} \PY{n}{child} \PY{o+ow}{in} \PY{n}{find}\PY{p}{:}
            \PY{n+nb}{print}\PY{p}{(}\PY{n}{child}\PY{o}{.}\PY{n}{tag}\PY{p}{,} \PY{l+s+s2}{\PYZdq{}}\PY{l+s+s2}{|}\PY{l+s+s2}{\PYZdq{}}\PY{p}{,} \PY{n}{child}\PY{o}{.}\PY{n}{text}\PY{p}{)}
            
        \PY{n}{find} \PY{o}{=} \PY{n}{tree}\PY{o}{.}\PY{n}{findall}\PY{p}{(}\PY{l+s+s1}{\PYZsq{}}\PY{l+s+s1}{.//dc:creatorName}\PY{l+s+s1}{\PYZsq{}}\PY{p}{,}\PY{n}{namespaces}\PY{p}{)}
        
        \PY{k}{for} \PY{n}{child} \PY{o+ow}{in} \PY{n}{find}\PY{p}{:}
            \PY{n+nb}{print}\PY{p}{(}\PY{n}{child}\PY{o}{.}\PY{n}{tag}\PY{p}{,} \PY{l+s+s2}{\PYZdq{}}\PY{l+s+s2}{|}\PY{l+s+s2}{\PYZdq{}}\PY{p}{,} \PY{n}{child}\PY{o}{.}\PY{n}{text}\PY{p}{)}
            
        \PY{n}{find} \PY{o}{=} \PY{n}{tree}\PY{o}{.}\PY{n}{findall}\PY{p}{(}\PY{l+s+s1}{\PYZsq{}}\PY{l+s+s1}{.//dc:subject}\PY{l+s+s1}{\PYZsq{}}\PY{p}{,}\PY{n}{namespaces}\PY{p}{)}
        
        \PY{k}{for} \PY{n}{child} \PY{o+ow}{in} \PY{n}{find}\PY{p}{:}
            \PY{n+nb}{print}\PY{p}{(}\PY{n}{child}\PY{o}{.}\PY{n}{tag}\PY{p}{,} \PY{l+s+s2}{\PYZdq{}}\PY{l+s+s2}{|}\PY{l+s+s2}{\PYZdq{}}\PY{p}{,} \PY{n}{child}\PY{o}{.}\PY{n}{text}\PY{p}{)}
        
        \PY{n}{find} \PY{o}{=} \PY{n}{tree}\PY{o}{.}\PY{n}{findall}\PY{p}{(}\PY{l+s+s1}{\PYZsq{}}\PY{l+s+s1}{.//dc:description}\PY{l+s+s1}{\PYZsq{}}\PY{p}{,}\PY{n}{namespaces}\PY{p}{)}
        
        \PY{k}{for} \PY{n}{child} \PY{o+ow}{in} \PY{n}{find}\PY{p}{:}
            \PY{n+nb}{print}\PY{p}{(}\PY{n}{child}\PY{o}{.}\PY{n}{tag}\PY{p}{,} \PY{l+s+s2}{\PYZdq{}}\PY{l+s+s2}{|}\PY{l+s+s2}{\PYZdq{}}\PY{p}{,} \PY{n}{child}\PY{o}{.}\PY{n}{text}\PY{p}{)}
            
        \PY{n}{find} \PY{o}{=} \PY{n}{tree}\PY{o}{.}\PY{n}{findall}\PY{p}{(}\PY{l+s+s1}{\PYZsq{}}\PY{l+s+s1}{.//dc:publisher}\PY{l+s+s1}{\PYZsq{}}\PY{p}{,}\PY{n}{namespaces}\PY{p}{)}
        
        \PY{k}{for} \PY{n}{child} \PY{o+ow}{in} \PY{n}{find}\PY{p}{:}
            \PY{n+nb}{print}\PY{p}{(}\PY{n}{child}\PY{o}{.}\PY{n}{tag}\PY{p}{,} \PY{l+s+s2}{\PYZdq{}}\PY{l+s+s2}{|}\PY{l+s+s2}{\PYZdq{}}\PY{p}{,} \PY{n}{child}\PY{o}{.}\PY{n}{text}\PY{p}{)}
            
        \PY{n}{find} \PY{o}{=} \PY{n}{tree}\PY{o}{.}\PY{n}{findall}\PY{p}{(}\PY{l+s+s1}{\PYZsq{}}\PY{l+s+s1}{.//dc:contributorName}\PY{l+s+s1}{\PYZsq{}}\PY{p}{,}\PY{n}{namespaces}\PY{p}{)}
        
        \PY{k}{for} \PY{n}{child} \PY{o+ow}{in} \PY{n}{find}\PY{p}{:}
            \PY{n+nb}{print}\PY{p}{(}\PY{n}{child}\PY{o}{.}\PY{n}{tag}\PY{p}{,} \PY{l+s+s2}{\PYZdq{}}\PY{l+s+s2}{|}\PY{l+s+s2}{\PYZdq{}}\PY{p}{,} \PY{n}{child}\PY{o}{.}\PY{n}{text}\PY{p}{)}
            
        \PY{n}{find} \PY{o}{=} \PY{n}{tree}\PY{o}{.}\PY{n}{findall}\PY{p}{(}\PY{l+s+s1}{\PYZsq{}}\PY{l+s+s1}{.//dc:date}\PY{l+s+s1}{\PYZsq{}}\PY{p}{,}\PY{n}{namespaces}\PY{p}{)}
        
        \PY{k}{for} \PY{n}{child} \PY{o+ow}{in} \PY{n}{find}\PY{p}{:}
            \PY{n+nb}{print}\PY{p}{(}\PY{n}{child}\PY{o}{.}\PY{n}{tag}\PY{p}{,} \PY{l+s+s2}{\PYZdq{}}\PY{l+s+s2}{|}\PY{l+s+s2}{\PYZdq{}}\PY{p}{,} \PY{n}{child}\PY{o}{.}\PY{n}{text}\PY{p}{)}
            
        \PY{n}{find} \PY{o}{=} \PY{n}{tree}\PY{o}{.}\PY{n}{findall}\PY{p}{(}\PY{l+s+s1}{\PYZsq{}}\PY{l+s+s1}{.//dc:resourceType}\PY{l+s+s1}{\PYZsq{}}\PY{p}{,}\PY{n}{namespaces}\PY{p}{)}
        
        \PY{k}{for} \PY{n}{child} \PY{o+ow}{in} \PY{n}{find}\PY{p}{:}
            \PY{n+nb}{print}\PY{p}{(}\PY{n}{child}\PY{o}{.}\PY{n}{tag}\PY{p}{,} \PY{l+s+s2}{\PYZdq{}}\PY{l+s+s2}{|}\PY{l+s+s2}{\PYZdq{}}\PY{p}{,} \PY{n}{child}\PY{o}{.}\PY{n}{text}\PY{p}{)}
            
        \PY{n}{find} \PY{o}{=} \PY{n}{tree}\PY{o}{.}\PY{n}{findall}\PY{p}{(}\PY{l+s+s1}{\PYZsq{}}\PY{l+s+s1}{.//dc:format}\PY{l+s+s1}{\PYZsq{}}\PY{p}{,}\PY{n}{namespaces}\PY{p}{)}
        
        \PY{k}{for} \PY{n}{child} \PY{o+ow}{in} \PY{n}{find}\PY{p}{:}
            \PY{n+nb}{print}\PY{p}{(}\PY{n}{child}\PY{o}{.}\PY{n}{tag}\PY{p}{,} \PY{l+s+s2}{\PYZdq{}}\PY{l+s+s2}{|}\PY{l+s+s2}{\PYZdq{}}\PY{p}{,} \PY{n}{child}\PY{o}{.}\PY{n}{text}\PY{p}{)}
            
        \PY{n}{find} \PY{o}{=} \PY{n}{tree}\PY{o}{.}\PY{n}{findall}\PY{p}{(}\PY{l+s+s1}{\PYZsq{}}\PY{l+s+s1}{.//dc:identifier}\PY{l+s+s1}{\PYZsq{}}\PY{p}{,}\PY{n}{namespaces}\PY{p}{)}
        
        \PY{k}{for} \PY{n}{child} \PY{o+ow}{in} \PY{n}{find}\PY{p}{:}
            \PY{n+nb}{print}\PY{p}{(}\PY{n}{child}\PY{o}{.}\PY{n}{tag}\PY{p}{,} \PY{l+s+s2}{\PYZdq{}}\PY{l+s+s2}{|}\PY{l+s+s2}{\PYZdq{}}\PY{p}{,} \PY{n}{child}\PY{o}{.}\PY{n}{text}\PY{p}{)}
            
        \PY{n+nb}{print}\PY{p}{(}\PY{l+s+s1}{\PYZsq{}}\PY{l+s+s1}{Source: NA}\PY{l+s+s1}{\PYZsq{}}\PY{p}{)}
        
        \PY{n}{find} \PY{o}{=} \PY{n}{tree}\PY{o}{.}\PY{n}{findall}\PY{p}{(}\PY{l+s+s1}{\PYZsq{}}\PY{l+s+s1}{.//dc:language}\PY{l+s+s1}{\PYZsq{}}\PY{p}{,}\PY{n}{namespaces}\PY{p}{)}
        
        \PY{k}{for} \PY{n}{child} \PY{o+ow}{in} \PY{n}{find}\PY{p}{:}
            \PY{n+nb}{print}\PY{p}{(}\PY{n}{child}\PY{o}{.}\PY{n}{tag}\PY{p}{,} \PY{l+s+s2}{\PYZdq{}}\PY{l+s+s2}{|}\PY{l+s+s2}{\PYZdq{}}\PY{p}{,} \PY{n}{child}\PY{o}{.}\PY{n}{text}\PY{p}{)}
        
        \PY{n}{find} \PY{o}{=} \PY{n}{tree}\PY{o}{.}\PY{n}{findall}\PY{p}{(}\PY{l+s+s1}{\PYZsq{}}\PY{l+s+s1}{.//dc:relatedIdentifier}\PY{l+s+s1}{\PYZsq{}}\PY{p}{,}\PY{n}{namespaces}\PY{p}{)}
        
        \PY{k}{for} \PY{n}{child} \PY{o+ow}{in} \PY{n}{find}\PY{p}{:}
            \PY{n+nb}{print}\PY{p}{(}\PY{n}{child}\PY{o}{.}\PY{n}{tag}\PY{p}{,} \PY{l+s+s2}{\PYZdq{}}\PY{l+s+s2}{|}\PY{l+s+s2}{\PYZdq{}}\PY{p}{,} \PY{n}{child}\PY{o}{.}\PY{n}{text}\PY{p}{)}
        
        \PY{n}{find1} \PY{o}{=} \PY{n}{tree}\PY{o}{.}\PY{n}{findall}\PY{p}{(}\PY{l+s+s1}{\PYZsq{}}\PY{l+s+s1}{.//dc:geoLocationPoint}\PY{l+s+s1}{\PYZsq{}}\PY{p}{,}\PY{n}{namespaces}\PY{p}{)} 
        \PY{n}{find2} \PY{o}{=} \PY{n}{tree}\PY{o}{.}\PY{n}{findall}\PY{p}{(}\PY{l+s+s1}{\PYZsq{}}\PY{l+s+s1}{.//dc:geoLocationPlace}\PY{l+s+s1}{\PYZsq{}}\PY{p}{,}\PY{n}{namespaces}\PY{p}{)}
        
        \PY{k}{for} \PY{n}{child1}\PY{p}{,} \PY{n}{child2} \PY{o+ow}{in} \PY{n+nb}{zip}\PY{p}{(}\PY{n}{find1}\PY{p}{,} \PY{n}{find2}\PY{p}{)}\PY{p}{:}
            \PY{n+nb}{print}\PY{p}{(}\PY{l+s+s1}{\PYZsq{}}\PY{l+s+s1}{\PYZob{}}\PY{l+s+s1}{\PYZsq{}}\PY{p}{,}\PY{n}{namespaces}\PY{p}{[}\PY{l+s+s1}{\PYZsq{}}\PY{l+s+s1}{dc}\PY{l+s+s1}{\PYZsq{}}\PY{p}{]}\PY{p}{,}\PY{l+s+s1}{\PYZsq{}}\PY{l+s+s1}{\PYZcb{}}\PY{l+s+s1}{\PYZsq{}}\PY{p}{,} \PY{l+s+s1}{\PYZsq{}}\PY{l+s+s1}{coverage}\PY{l+s+s1}{\PYZsq{}}\PY{p}{,} \PY{l+s+s2}{\PYZdq{}}\PY{l+s+s2}{|}\PY{l+s+s2}{\PYZdq{}}\PY{p}{,} \PY{n}{child1}\PY{o}{.}\PY{n}{text} \PY{o}{+} \PY{l+s+s1}{\PYZsq{}}\PY{l+s+s1}{, }\PY{l+s+s1}{\PYZsq{}} \PY{o}{+} \PY{n}{child2}\PY{o}{.}\PY{n}{text}\PY{p}{)}
        
        \PY{k}{for} \PY{n}{child} \PY{o+ow}{in} \PY{n}{find}\PY{p}{:}
            \PY{n+nb}{print}\PY{p}{(}\PY{n}{child}\PY{o}{.}\PY{n}{tag}\PY{p}{,} \PY{l+s+s2}{\PYZdq{}}\PY{l+s+s2}{|}\PY{l+s+s2}{\PYZdq{}}\PY{p}{,} \PY{n}{child}\PY{o}{.}\PY{n}{text}\PY{p}{)}
        
        \PY{n}{find} \PY{o}{=} \PY{n}{tree}\PY{o}{.}\PY{n}{findall}\PY{p}{(}\PY{l+s+s1}{\PYZsq{}}\PY{l+s+s1}{.//dc:rights}\PY{l+s+s1}{\PYZsq{}}\PY{p}{,}\PY{n}{namespaces}\PY{p}{)}
        
        \PY{k}{for} \PY{n}{child} \PY{o+ow}{in} \PY{n}{find}\PY{p}{:}
            \PY{n+nb}{print}\PY{p}{(}\PY{n}{child}\PY{o}{.}\PY{n}{tag}\PY{p}{,} \PY{l+s+s2}{\PYZdq{}}\PY{l+s+s2}{|}\PY{l+s+s2}{\PYZdq{}}\PY{p}{,} \PY{n}{child}\PY{o}{.}\PY{n}{text}\PY{p}{)}
\end{Verbatim}


    \begin{Verbatim}[commandchars=\\\{\}]
\{http://datacite.org/schema/kernel-3\}title | Full DataCite XML Example
\{http://datacite.org/schema/kernel-3\}title | Demonstration of DataCite Properties.
\{http://datacite.org/schema/kernel-3\}creatorName | Miller, Elizabeth
\{http://datacite.org/schema/kernel-3\}subject | 000 computer science
\{http://datacite.org/schema/kernel-3\}description | 
            XML example of all DataCite Metadata Schema v3.1 properties.
        
\{http://datacite.org/schema/kernel-3\}publisher | DataCite
\{http://datacite.org/schema/kernel-3\}contributorName | Starr, Joan
\{http://datacite.org/schema/kernel-3\}date | 2014-10-17
\{http://datacite.org/schema/kernel-3\}resourceType | XML
\{http://datacite.org/schema/kernel-3\}format | application/xml
\{http://datacite.org/schema/kernel-3\}identifier | 10.5072/example-full
Source: NA
\{http://datacite.org/schema/kernel-3\}language | en-us
\{http://datacite.org/schema/kernel-3\}relatedIdentifier | http://data.datacite.org/application/citeproc+json/10.5072/example-full
\{http://datacite.org/schema/kernel-3\}relatedIdentifier | arXiv:0706.0001
\{ http://datacite.org/schema/kernel-3 \} coverage | 31.233 -67.302, Atlantic Ocean
\{http://datacite.org/schema/kernel-3\}relatedIdentifier | http://data.datacite.org/application/citeproc+json/10.5072/example-full
\{http://datacite.org/schema/kernel-3\}relatedIdentifier | arXiv:0706.0001
\{http://datacite.org/schema/kernel-3\}rights | CC0 1.0 Universal

    \end{Verbatim}

    \hypertarget{ejercicio-2}{%
\subsection{Ejercicio 2}\label{ejercicio-2}}

    Haz un listado de todas las etiquetas del documento XML con sus
atributos (si lo tienen)

    \begin{Verbatim}[commandchars=\\\{\}]
{\color{incolor}In [{\color{incolor}8}]:} \PY{n}{tree} \PY{o}{=} \PY{n}{ET}\PY{o}{.}\PY{n}{parse}\PY{p}{(}\PY{l+s+s1}{\PYZsq{}}\PY{l+s+s1}{datacite\PYZhy{}example\PYZhy{}full\PYZhy{}v3.1.xml}\PY{l+s+s1}{\PYZsq{}}\PY{p}{)}
        \PY{n}{root} \PY{o}{=} \PY{n}{tree}\PY{o}{.}\PY{n}{getroot}\PY{p}{(}\PY{p}{)}
        
        \PY{k}{for} \PY{n}{table} \PY{o+ow}{in} \PY{n}{root}\PY{o}{.}\PY{n}{getiterator}\PY{p}{(}\PY{p}{)}\PY{p}{:}
            \PY{k}{for} \PY{n}{child} \PY{o+ow}{in} \PY{n}{table}\PY{p}{:}
                \PY{k}{if} \PY{n}{child}\PY{o}{.}\PY{n}{attrib} \PY{o}{!=} \PY{p}{\PYZob{}}\PY{p}{\PYZcb{}}\PY{p}{:}
                    \PY{n+nb}{print}\PY{p}{(}\PY{n}{child}\PY{o}{.}\PY{n}{attrib}\PY{p}{,} \PY{l+s+s1}{\PYZsq{}}\PY{l+s+se}{\PYZbs{}n}\PY{l+s+s1}{\PYZsq{}}\PY{p}{,} \PY{n}{child}\PY{o}{.}\PY{n}{tag}\PY{p}{,} \PY{l+s+s1}{\PYZsq{}}\PY{l+s+se}{\PYZbs{}n}\PY{l+s+s1}{\PYZsq{}}\PY{p}{)}
                \PY{k}{else}\PY{p}{:}
                    \PY{n+nb}{print}\PY{p}{(}\PY{n}{child}\PY{o}{.}\PY{n}{tag}\PY{p}{,} \PY{l+s+s1}{\PYZsq{}}\PY{l+s+se}{\PYZbs{}n}\PY{l+s+s1}{\PYZsq{}}\PY{p}{)}
\end{Verbatim}


    \begin{Verbatim}[commandchars=\\\{\}]
\{'identifierType': 'DOI'\} 
 \{http://datacite.org/schema/kernel-3\}identifier 

\{http://datacite.org/schema/kernel-3\}creators 

\{http://datacite.org/schema/kernel-3\}titles 

\{http://datacite.org/schema/kernel-3\}publisher 

\{http://datacite.org/schema/kernel-3\}publicationYear 

\{http://datacite.org/schema/kernel-3\}subjects 

\{http://datacite.org/schema/kernel-3\}contributors 

\{http://datacite.org/schema/kernel-3\}dates 

\{http://datacite.org/schema/kernel-3\}language 

\{'resourceTypeGeneral': 'Software'\} 
 \{http://datacite.org/schema/kernel-3\}resourceType 

\{http://datacite.org/schema/kernel-3\}alternateIdentifiers 

\{http://datacite.org/schema/kernel-3\}relatedIdentifiers 

\{http://datacite.org/schema/kernel-3\}sizes 

\{http://datacite.org/schema/kernel-3\}formats 

\{http://datacite.org/schema/kernel-3\}version 

\{http://datacite.org/schema/kernel-3\}rightsList 

\{http://datacite.org/schema/kernel-3\}descriptions 

\{http://datacite.org/schema/kernel-3\}geoLocations 

\{http://datacite.org/schema/kernel-3\}creator 

\{http://datacite.org/schema/kernel-3\}creatorName 

\{'schemeURI': 'http://orcid.org/', 'nameIdentifierScheme': 'ORCID'\} 
 \{http://datacite.org/schema/kernel-3\}nameIdentifier 

\{http://datacite.org/schema/kernel-3\}affiliation 

\{'\{http://www.w3.org/XML/1998/namespace\}lang': 'en-us'\} 
 \{http://datacite.org/schema/kernel-3\}title 

\{'\{http://www.w3.org/XML/1998/namespace\}lang': 'en-us', 'titleType': 'Subtitle'\} 
 \{http://datacite.org/schema/kernel-3\}title 

\{'\{http://www.w3.org/XML/1998/namespace\}lang': 'en-us', 'schemeURI': 'http://dewey.info/', 'subjectScheme': 'dewey'\} 
 \{http://datacite.org/schema/kernel-3\}subject 

\{'contributorType': 'ProjectLeader'\} 
 \{http://datacite.org/schema/kernel-3\}contributor 

\{http://datacite.org/schema/kernel-3\}contributorName 

\{'schemeURI': 'http://orcid.org/', 'nameIdentifierScheme': 'ORCID'\} 
 \{http://datacite.org/schema/kernel-3\}nameIdentifier 

\{http://datacite.org/schema/kernel-3\}affiliation 

\{'dateType': 'Updated'\} 
 \{http://datacite.org/schema/kernel-3\}date 

\{'alternateIdentifierType': 'URL'\} 
 \{http://datacite.org/schema/kernel-3\}alternateIdentifier 

\{'relatedIdentifierType': 'URL', 'relationType': 'HasMetadata', 'relatedMetadataScheme': 'citeproc+json', 'schemeURI': 'https://github.com/citation-style-language/schema/raw/master/csl-data.json'\} 
 \{http://datacite.org/schema/kernel-3\}relatedIdentifier 

\{'relatedIdentifierType': 'arXiv', 'relationType': 'IsReviewedBy'\} 
 \{http://datacite.org/schema/kernel-3\}relatedIdentifier 

\{http://datacite.org/schema/kernel-3\}size 

\{http://datacite.org/schema/kernel-3\}format 

\{'rightsURI': 'http://creativecommons.org/publicdomain/zero/1.0/'\} 
 \{http://datacite.org/schema/kernel-3\}rights 

\{'\{http://www.w3.org/XML/1998/namespace\}lang': 'en-us', 'descriptionType': 'Abstract'\} 
 \{http://datacite.org/schema/kernel-3\}description 

\{http://datacite.org/schema/kernel-3\}geoLocation 

\{http://datacite.org/schema/kernel-3\}geoLocationPoint 

\{http://datacite.org/schema/kernel-3\}geoLocationBox 

\{http://datacite.org/schema/kernel-3\}geoLocationPlace 


    \end{Verbatim}

    \hypertarget{ejercicio-3}{%
\subsection{Ejercicio 3}\label{ejercicio-3}}

    Muestra los distintos identificadores que tiene ese documento de este
modo: Identificador {[}tipo{]} = {[}identificador{]}

Ejemplo: Identificador DOI = 10.3122/121321

    \begin{Verbatim}[commandchars=\\\{\}]
{\color{incolor}In [{\color{incolor}10}]:} \PY{n}{tree} \PY{o}{=} \PY{n}{ET}\PY{o}{.}\PY{n}{parse}\PY{p}{(}\PY{l+s+s1}{\PYZsq{}}\PY{l+s+s1}{datacite\PYZhy{}example\PYZhy{}full\PYZhy{}v3.1.xml}\PY{l+s+s1}{\PYZsq{}}\PY{p}{)}
         \PY{n}{namespaces} \PY{o}{=} \PY{p}{\PYZob{}}\PY{l+s+s1}{\PYZsq{}}\PY{l+s+s1}{dc}\PY{l+s+s1}{\PYZsq{}}\PY{p}{:} \PY{l+s+s1}{\PYZsq{}}\PY{l+s+s1}{http://datacite.org/schema/kernel\PYZhy{}3}\PY{l+s+s1}{\PYZsq{}}\PY{p}{\PYZcb{}}
         \PY{n}{find} \PY{o}{=} \PY{n}{tree}\PY{o}{.}\PY{n}{findall}\PY{p}{(}\PY{l+s+s1}{\PYZsq{}}\PY{l+s+s1}{.//dc:identifier}\PY{l+s+s1}{\PYZsq{}}\PY{p}{,} \PY{n}{namespaces}\PY{p}{)}
         \PY{k}{for} \PY{n}{elem} \PY{o+ow}{in} \PY{n}{find}\PY{p}{:}
             \PY{n+nb}{print}\PY{p}{(}\PY{l+s+s1}{\PYZsq{}}\PY{l+s+s1}{Identificador [}\PY{l+s+s1}{\PYZsq{}}\PY{p}{,} \PY{n}{elem}\PY{o}{.}\PY{n}{attrib}\PY{p}{[}\PY{l+s+s1}{\PYZsq{}}\PY{l+s+s1}{identifierType}\PY{l+s+s1}{\PYZsq{}}\PY{p}{]}\PY{p}{,} \PY{l+s+s1}{\PYZsq{}}\PY{l+s+s1}{] = [}\PY{l+s+s1}{\PYZsq{}}\PY{p}{,} \PY{n}{elem}\PY{o}{.}\PY{n}{text}\PY{p}{,} \PY{l+s+s1}{\PYZsq{}}\PY{l+s+s1}{]}\PY{l+s+s1}{\PYZsq{}}\PY{p}{)}
         
         \PY{n}{find} \PY{o}{=} \PY{n}{tree}\PY{o}{.}\PY{n}{findall}\PY{p}{(}\PY{l+s+s1}{\PYZsq{}}\PY{l+s+s1}{.//dc:alternateIdentifier}\PY{l+s+s1}{\PYZsq{}}\PY{p}{,} \PY{n}{namespaces}\PY{p}{)}
         \PY{k}{for} \PY{n}{elem} \PY{o+ow}{in} \PY{n}{find}\PY{p}{:}
             \PY{n+nb}{print}\PY{p}{(}\PY{l+s+s1}{\PYZsq{}}\PY{l+s+s1}{Identificador [}\PY{l+s+s1}{\PYZsq{}}\PY{p}{,} \PY{n}{elem}\PY{o}{.}\PY{n}{attrib}\PY{p}{[}\PY{l+s+s1}{\PYZsq{}}\PY{l+s+s1}{alternateIdentifierType}\PY{l+s+s1}{\PYZsq{}}\PY{p}{]}\PY{p}{,} \PY{l+s+s1}{\PYZsq{}}\PY{l+s+s1}{] = [}\PY{l+s+s1}{\PYZsq{}}\PY{p}{,} \PY{n}{elem}\PY{o}{.}\PY{n}{text}\PY{p}{,} \PY{l+s+s1}{\PYZsq{}}\PY{l+s+s1}{]}\PY{l+s+s1}{\PYZsq{}}\PY{p}{)}
\end{Verbatim}


    \begin{Verbatim}[commandchars=\\\{\}]
Identificador [ DOI ] = [ 10.5072/example-full ]
Identificador [ URL ] = [ http://schema.datacite.org/schema/meta/kernel-3.1/example/datacite-example-full-v3.1.xml ]

    \end{Verbatim}


    % Add a bibliography block to the postdoc
    
    
    
    \end{document}
