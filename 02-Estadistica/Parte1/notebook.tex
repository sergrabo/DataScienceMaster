
% Default to the notebook output style

    


% Inherit from the specified cell style.




    
\documentclass[11pt]{article}

    
    
    \usepackage[T1]{fontenc}
    % Nicer default font (+ math font) than Computer Modern for most use cases
    \usepackage{mathpazo}

    % Basic figure setup, for now with no caption control since it's done
    % automatically by Pandoc (which extracts ![](path) syntax from Markdown).
    \usepackage{graphicx}
    % We will generate all images so they have a width \maxwidth. This means
    % that they will get their normal width if they fit onto the page, but
    % are scaled down if they would overflow the margins.
    \makeatletter
    \def\maxwidth{\ifdim\Gin@nat@width>\linewidth\linewidth
    \else\Gin@nat@width\fi}
    \makeatother
    \let\Oldincludegraphics\includegraphics
    % Set max figure width to be 80% of text width, for now hardcoded.
    \renewcommand{\includegraphics}[1]{\Oldincludegraphics[width=.8\maxwidth]{#1}}
    % Ensure that by default, figures have no caption (until we provide a
    % proper Figure object with a Caption API and a way to capture that
    % in the conversion process - todo).
    \usepackage{caption}
    \DeclareCaptionLabelFormat{nolabel}{}
    \captionsetup{labelformat=nolabel}

    \usepackage{adjustbox} % Used to constrain images to a maximum size 
    \usepackage{xcolor} % Allow colors to be defined
    \usepackage{enumerate} % Needed for markdown enumerations to work
    \usepackage{geometry} % Used to adjust the document margins
    \usepackage{amsmath} % Equations
    \usepackage{amssymb} % Equations
    \usepackage{textcomp} % defines textquotesingle
    % Hack from http://tex.stackexchange.com/a/47451/13684:
    \AtBeginDocument{%
        \def\PYZsq{\textquotesingle}% Upright quotes in Pygmentized code
    }
    \usepackage{upquote} % Upright quotes for verbatim code
    \usepackage{eurosym} % defines \euro
    \usepackage[mathletters]{ucs} % Extended unicode (utf-8) support
    \usepackage[utf8x]{inputenc} % Allow utf-8 characters in the tex document
    \usepackage{fancyvrb} % verbatim replacement that allows latex
    \usepackage{grffile} % extends the file name processing of package graphics 
                         % to support a larger range 
    % The hyperref package gives us a pdf with properly built
    % internal navigation ('pdf bookmarks' for the table of contents,
    % internal cross-reference links, web links for URLs, etc.)
    \usepackage{hyperref}
    \usepackage{longtable} % longtable support required by pandoc >1.10
    \usepackage{booktabs}  % table support for pandoc > 1.12.2
    \usepackage[inline]{enumitem} % IRkernel/repr support (it uses the enumerate* environment)
    \usepackage[normalem]{ulem} % ulem is needed to support strikethroughs (\sout)
                                % normalem makes italics be italics, not underlines
    

    
    
    % Colors for the hyperref package
    \definecolor{urlcolor}{rgb}{0,.145,.698}
    \definecolor{linkcolor}{rgb}{.71,0.21,0.01}
    \definecolor{citecolor}{rgb}{.12,.54,.11}

    % ANSI colors
    \definecolor{ansi-black}{HTML}{3E424D}
    \definecolor{ansi-black-intense}{HTML}{282C36}
    \definecolor{ansi-red}{HTML}{E75C58}
    \definecolor{ansi-red-intense}{HTML}{B22B31}
    \definecolor{ansi-green}{HTML}{00A250}
    \definecolor{ansi-green-intense}{HTML}{007427}
    \definecolor{ansi-yellow}{HTML}{DDB62B}
    \definecolor{ansi-yellow-intense}{HTML}{B27D12}
    \definecolor{ansi-blue}{HTML}{208FFB}
    \definecolor{ansi-blue-intense}{HTML}{0065CA}
    \definecolor{ansi-magenta}{HTML}{D160C4}
    \definecolor{ansi-magenta-intense}{HTML}{A03196}
    \definecolor{ansi-cyan}{HTML}{60C6C8}
    \definecolor{ansi-cyan-intense}{HTML}{258F8F}
    \definecolor{ansi-white}{HTML}{C5C1B4}
    \definecolor{ansi-white-intense}{HTML}{A1A6B2}

    % commands and environments needed by pandoc snippets
    % extracted from the output of `pandoc -s`
    \providecommand{\tightlist}{%
      \setlength{\itemsep}{0pt}\setlength{\parskip}{0pt}}
    \DefineVerbatimEnvironment{Highlighting}{Verbatim}{commandchars=\\\{\}}
    % Add ',fontsize=\small' for more characters per line
    \newenvironment{Shaded}{}{}
    \newcommand{\KeywordTok}[1]{\textcolor[rgb]{0.00,0.44,0.13}{\textbf{{#1}}}}
    \newcommand{\DataTypeTok}[1]{\textcolor[rgb]{0.56,0.13,0.00}{{#1}}}
    \newcommand{\DecValTok}[1]{\textcolor[rgb]{0.25,0.63,0.44}{{#1}}}
    \newcommand{\BaseNTok}[1]{\textcolor[rgb]{0.25,0.63,0.44}{{#1}}}
    \newcommand{\FloatTok}[1]{\textcolor[rgb]{0.25,0.63,0.44}{{#1}}}
    \newcommand{\CharTok}[1]{\textcolor[rgb]{0.25,0.44,0.63}{{#1}}}
    \newcommand{\StringTok}[1]{\textcolor[rgb]{0.25,0.44,0.63}{{#1}}}
    \newcommand{\CommentTok}[1]{\textcolor[rgb]{0.38,0.63,0.69}{\textit{{#1}}}}
    \newcommand{\OtherTok}[1]{\textcolor[rgb]{0.00,0.44,0.13}{{#1}}}
    \newcommand{\AlertTok}[1]{\textcolor[rgb]{1.00,0.00,0.00}{\textbf{{#1}}}}
    \newcommand{\FunctionTok}[1]{\textcolor[rgb]{0.02,0.16,0.49}{{#1}}}
    \newcommand{\RegionMarkerTok}[1]{{#1}}
    \newcommand{\ErrorTok}[1]{\textcolor[rgb]{1.00,0.00,0.00}{\textbf{{#1}}}}
    \newcommand{\NormalTok}[1]{{#1}}
    
    % Additional commands for more recent versions of Pandoc
    \newcommand{\ConstantTok}[1]{\textcolor[rgb]{0.53,0.00,0.00}{{#1}}}
    \newcommand{\SpecialCharTok}[1]{\textcolor[rgb]{0.25,0.44,0.63}{{#1}}}
    \newcommand{\VerbatimStringTok}[1]{\textcolor[rgb]{0.25,0.44,0.63}{{#1}}}
    \newcommand{\SpecialStringTok}[1]{\textcolor[rgb]{0.73,0.40,0.53}{{#1}}}
    \newcommand{\ImportTok}[1]{{#1}}
    \newcommand{\DocumentationTok}[1]{\textcolor[rgb]{0.73,0.13,0.13}{\textit{{#1}}}}
    \newcommand{\AnnotationTok}[1]{\textcolor[rgb]{0.38,0.63,0.69}{\textbf{\textit{{#1}}}}}
    \newcommand{\CommentVarTok}[1]{\textcolor[rgb]{0.38,0.63,0.69}{\textbf{\textit{{#1}}}}}
    \newcommand{\VariableTok}[1]{\textcolor[rgb]{0.10,0.09,0.49}{{#1}}}
    \newcommand{\ControlFlowTok}[1]{\textcolor[rgb]{0.00,0.44,0.13}{\textbf{{#1}}}}
    \newcommand{\OperatorTok}[1]{\textcolor[rgb]{0.40,0.40,0.40}{{#1}}}
    \newcommand{\BuiltInTok}[1]{{#1}}
    \newcommand{\ExtensionTok}[1]{{#1}}
    \newcommand{\PreprocessorTok}[1]{\textcolor[rgb]{0.74,0.48,0.00}{{#1}}}
    \newcommand{\AttributeTok}[1]{\textcolor[rgb]{0.49,0.56,0.16}{{#1}}}
    \newcommand{\InformationTok}[1]{\textcolor[rgb]{0.38,0.63,0.69}{\textbf{\textit{{#1}}}}}
    \newcommand{\WarningTok}[1]{\textcolor[rgb]{0.38,0.63,0.69}{\textbf{\textit{{#1}}}}}
    
    
    % Define a nice break command that doesn't care if a line doesn't already
    % exist.
    \def\br{\hspace*{\fill} \\* }
    % Math Jax compatability definitions
    \def\gt{>}
    \def\lt{<}
    % Document parameters
    \title{P1}
    
    
    

    % Pygments definitions
    
\makeatletter
\def\PY@reset{\let\PY@it=\relax \let\PY@bf=\relax%
    \let\PY@ul=\relax \let\PY@tc=\relax%
    \let\PY@bc=\relax \let\PY@ff=\relax}
\def\PY@tok#1{\csname PY@tok@#1\endcsname}
\def\PY@toks#1+{\ifx\relax#1\empty\else%
    \PY@tok{#1}\expandafter\PY@toks\fi}
\def\PY@do#1{\PY@bc{\PY@tc{\PY@ul{%
    \PY@it{\PY@bf{\PY@ff{#1}}}}}}}
\def\PY#1#2{\PY@reset\PY@toks#1+\relax+\PY@do{#2}}

\expandafter\def\csname PY@tok@w\endcsname{\def\PY@tc##1{\textcolor[rgb]{0.73,0.73,0.73}{##1}}}
\expandafter\def\csname PY@tok@c\endcsname{\let\PY@it=\textit\def\PY@tc##1{\textcolor[rgb]{0.25,0.50,0.50}{##1}}}
\expandafter\def\csname PY@tok@cp\endcsname{\def\PY@tc##1{\textcolor[rgb]{0.74,0.48,0.00}{##1}}}
\expandafter\def\csname PY@tok@k\endcsname{\let\PY@bf=\textbf\def\PY@tc##1{\textcolor[rgb]{0.00,0.50,0.00}{##1}}}
\expandafter\def\csname PY@tok@kp\endcsname{\def\PY@tc##1{\textcolor[rgb]{0.00,0.50,0.00}{##1}}}
\expandafter\def\csname PY@tok@kt\endcsname{\def\PY@tc##1{\textcolor[rgb]{0.69,0.00,0.25}{##1}}}
\expandafter\def\csname PY@tok@o\endcsname{\def\PY@tc##1{\textcolor[rgb]{0.40,0.40,0.40}{##1}}}
\expandafter\def\csname PY@tok@ow\endcsname{\let\PY@bf=\textbf\def\PY@tc##1{\textcolor[rgb]{0.67,0.13,1.00}{##1}}}
\expandafter\def\csname PY@tok@nb\endcsname{\def\PY@tc##1{\textcolor[rgb]{0.00,0.50,0.00}{##1}}}
\expandafter\def\csname PY@tok@nf\endcsname{\def\PY@tc##1{\textcolor[rgb]{0.00,0.00,1.00}{##1}}}
\expandafter\def\csname PY@tok@nc\endcsname{\let\PY@bf=\textbf\def\PY@tc##1{\textcolor[rgb]{0.00,0.00,1.00}{##1}}}
\expandafter\def\csname PY@tok@nn\endcsname{\let\PY@bf=\textbf\def\PY@tc##1{\textcolor[rgb]{0.00,0.00,1.00}{##1}}}
\expandafter\def\csname PY@tok@ne\endcsname{\let\PY@bf=\textbf\def\PY@tc##1{\textcolor[rgb]{0.82,0.25,0.23}{##1}}}
\expandafter\def\csname PY@tok@nv\endcsname{\def\PY@tc##1{\textcolor[rgb]{0.10,0.09,0.49}{##1}}}
\expandafter\def\csname PY@tok@no\endcsname{\def\PY@tc##1{\textcolor[rgb]{0.53,0.00,0.00}{##1}}}
\expandafter\def\csname PY@tok@nl\endcsname{\def\PY@tc##1{\textcolor[rgb]{0.63,0.63,0.00}{##1}}}
\expandafter\def\csname PY@tok@ni\endcsname{\let\PY@bf=\textbf\def\PY@tc##1{\textcolor[rgb]{0.60,0.60,0.60}{##1}}}
\expandafter\def\csname PY@tok@na\endcsname{\def\PY@tc##1{\textcolor[rgb]{0.49,0.56,0.16}{##1}}}
\expandafter\def\csname PY@tok@nt\endcsname{\let\PY@bf=\textbf\def\PY@tc##1{\textcolor[rgb]{0.00,0.50,0.00}{##1}}}
\expandafter\def\csname PY@tok@nd\endcsname{\def\PY@tc##1{\textcolor[rgb]{0.67,0.13,1.00}{##1}}}
\expandafter\def\csname PY@tok@s\endcsname{\def\PY@tc##1{\textcolor[rgb]{0.73,0.13,0.13}{##1}}}
\expandafter\def\csname PY@tok@sd\endcsname{\let\PY@it=\textit\def\PY@tc##1{\textcolor[rgb]{0.73,0.13,0.13}{##1}}}
\expandafter\def\csname PY@tok@si\endcsname{\let\PY@bf=\textbf\def\PY@tc##1{\textcolor[rgb]{0.73,0.40,0.53}{##1}}}
\expandafter\def\csname PY@tok@se\endcsname{\let\PY@bf=\textbf\def\PY@tc##1{\textcolor[rgb]{0.73,0.40,0.13}{##1}}}
\expandafter\def\csname PY@tok@sr\endcsname{\def\PY@tc##1{\textcolor[rgb]{0.73,0.40,0.53}{##1}}}
\expandafter\def\csname PY@tok@ss\endcsname{\def\PY@tc##1{\textcolor[rgb]{0.10,0.09,0.49}{##1}}}
\expandafter\def\csname PY@tok@sx\endcsname{\def\PY@tc##1{\textcolor[rgb]{0.00,0.50,0.00}{##1}}}
\expandafter\def\csname PY@tok@m\endcsname{\def\PY@tc##1{\textcolor[rgb]{0.40,0.40,0.40}{##1}}}
\expandafter\def\csname PY@tok@gh\endcsname{\let\PY@bf=\textbf\def\PY@tc##1{\textcolor[rgb]{0.00,0.00,0.50}{##1}}}
\expandafter\def\csname PY@tok@gu\endcsname{\let\PY@bf=\textbf\def\PY@tc##1{\textcolor[rgb]{0.50,0.00,0.50}{##1}}}
\expandafter\def\csname PY@tok@gd\endcsname{\def\PY@tc##1{\textcolor[rgb]{0.63,0.00,0.00}{##1}}}
\expandafter\def\csname PY@tok@gi\endcsname{\def\PY@tc##1{\textcolor[rgb]{0.00,0.63,0.00}{##1}}}
\expandafter\def\csname PY@tok@gr\endcsname{\def\PY@tc##1{\textcolor[rgb]{1.00,0.00,0.00}{##1}}}
\expandafter\def\csname PY@tok@ge\endcsname{\let\PY@it=\textit}
\expandafter\def\csname PY@tok@gs\endcsname{\let\PY@bf=\textbf}
\expandafter\def\csname PY@tok@gp\endcsname{\let\PY@bf=\textbf\def\PY@tc##1{\textcolor[rgb]{0.00,0.00,0.50}{##1}}}
\expandafter\def\csname PY@tok@go\endcsname{\def\PY@tc##1{\textcolor[rgb]{0.53,0.53,0.53}{##1}}}
\expandafter\def\csname PY@tok@gt\endcsname{\def\PY@tc##1{\textcolor[rgb]{0.00,0.27,0.87}{##1}}}
\expandafter\def\csname PY@tok@err\endcsname{\def\PY@bc##1{\setlength{\fboxsep}{0pt}\fcolorbox[rgb]{1.00,0.00,0.00}{1,1,1}{\strut ##1}}}
\expandafter\def\csname PY@tok@kc\endcsname{\let\PY@bf=\textbf\def\PY@tc##1{\textcolor[rgb]{0.00,0.50,0.00}{##1}}}
\expandafter\def\csname PY@tok@kd\endcsname{\let\PY@bf=\textbf\def\PY@tc##1{\textcolor[rgb]{0.00,0.50,0.00}{##1}}}
\expandafter\def\csname PY@tok@kn\endcsname{\let\PY@bf=\textbf\def\PY@tc##1{\textcolor[rgb]{0.00,0.50,0.00}{##1}}}
\expandafter\def\csname PY@tok@kr\endcsname{\let\PY@bf=\textbf\def\PY@tc##1{\textcolor[rgb]{0.00,0.50,0.00}{##1}}}
\expandafter\def\csname PY@tok@bp\endcsname{\def\PY@tc##1{\textcolor[rgb]{0.00,0.50,0.00}{##1}}}
\expandafter\def\csname PY@tok@fm\endcsname{\def\PY@tc##1{\textcolor[rgb]{0.00,0.00,1.00}{##1}}}
\expandafter\def\csname PY@tok@vc\endcsname{\def\PY@tc##1{\textcolor[rgb]{0.10,0.09,0.49}{##1}}}
\expandafter\def\csname PY@tok@vg\endcsname{\def\PY@tc##1{\textcolor[rgb]{0.10,0.09,0.49}{##1}}}
\expandafter\def\csname PY@tok@vi\endcsname{\def\PY@tc##1{\textcolor[rgb]{0.10,0.09,0.49}{##1}}}
\expandafter\def\csname PY@tok@vm\endcsname{\def\PY@tc##1{\textcolor[rgb]{0.10,0.09,0.49}{##1}}}
\expandafter\def\csname PY@tok@sa\endcsname{\def\PY@tc##1{\textcolor[rgb]{0.73,0.13,0.13}{##1}}}
\expandafter\def\csname PY@tok@sb\endcsname{\def\PY@tc##1{\textcolor[rgb]{0.73,0.13,0.13}{##1}}}
\expandafter\def\csname PY@tok@sc\endcsname{\def\PY@tc##1{\textcolor[rgb]{0.73,0.13,0.13}{##1}}}
\expandafter\def\csname PY@tok@dl\endcsname{\def\PY@tc##1{\textcolor[rgb]{0.73,0.13,0.13}{##1}}}
\expandafter\def\csname PY@tok@s2\endcsname{\def\PY@tc##1{\textcolor[rgb]{0.73,0.13,0.13}{##1}}}
\expandafter\def\csname PY@tok@sh\endcsname{\def\PY@tc##1{\textcolor[rgb]{0.73,0.13,0.13}{##1}}}
\expandafter\def\csname PY@tok@s1\endcsname{\def\PY@tc##1{\textcolor[rgb]{0.73,0.13,0.13}{##1}}}
\expandafter\def\csname PY@tok@mb\endcsname{\def\PY@tc##1{\textcolor[rgb]{0.40,0.40,0.40}{##1}}}
\expandafter\def\csname PY@tok@mf\endcsname{\def\PY@tc##1{\textcolor[rgb]{0.40,0.40,0.40}{##1}}}
\expandafter\def\csname PY@tok@mh\endcsname{\def\PY@tc##1{\textcolor[rgb]{0.40,0.40,0.40}{##1}}}
\expandafter\def\csname PY@tok@mi\endcsname{\def\PY@tc##1{\textcolor[rgb]{0.40,0.40,0.40}{##1}}}
\expandafter\def\csname PY@tok@il\endcsname{\def\PY@tc##1{\textcolor[rgb]{0.40,0.40,0.40}{##1}}}
\expandafter\def\csname PY@tok@mo\endcsname{\def\PY@tc##1{\textcolor[rgb]{0.40,0.40,0.40}{##1}}}
\expandafter\def\csname PY@tok@ch\endcsname{\let\PY@it=\textit\def\PY@tc##1{\textcolor[rgb]{0.25,0.50,0.50}{##1}}}
\expandafter\def\csname PY@tok@cm\endcsname{\let\PY@it=\textit\def\PY@tc##1{\textcolor[rgb]{0.25,0.50,0.50}{##1}}}
\expandafter\def\csname PY@tok@cpf\endcsname{\let\PY@it=\textit\def\PY@tc##1{\textcolor[rgb]{0.25,0.50,0.50}{##1}}}
\expandafter\def\csname PY@tok@c1\endcsname{\let\PY@it=\textit\def\PY@tc##1{\textcolor[rgb]{0.25,0.50,0.50}{##1}}}
\expandafter\def\csname PY@tok@cs\endcsname{\let\PY@it=\textit\def\PY@tc##1{\textcolor[rgb]{0.25,0.50,0.50}{##1}}}

\def\PYZbs{\char`\\}
\def\PYZus{\char`\_}
\def\PYZob{\char`\{}
\def\PYZcb{\char`\}}
\def\PYZca{\char`\^}
\def\PYZam{\char`\&}
\def\PYZlt{\char`\<}
\def\PYZgt{\char`\>}
\def\PYZsh{\char`\#}
\def\PYZpc{\char`\%}
\def\PYZdl{\char`\$}
\def\PYZhy{\char`\-}
\def\PYZsq{\char`\'}
\def\PYZdq{\char`\"}
\def\PYZti{\char`\~}
% for compatibility with earlier versions
\def\PYZat{@}
\def\PYZlb{[}
\def\PYZrb{]}
\makeatother


    % Exact colors from NB
    \definecolor{incolor}{rgb}{0.0, 0.0, 0.5}
    \definecolor{outcolor}{rgb}{0.545, 0.0, 0.0}



    
    % Prevent overflowing lines due to hard-to-break entities
    \sloppy 
    % Setup hyperref package
    \hypersetup{
      breaklinks=true,  % so long urls are correctly broken across lines
      colorlinks=true,
      urlcolor=urlcolor,
      linkcolor=linkcolor,
      citecolor=citecolor,
      }
    % Slightly bigger margins than the latex defaults
    
    \geometry{verbose,tmargin=1in,bmargin=1in,lmargin=1in,rmargin=1in}
    
    

    \begin{document}
    
    
    \maketitle
    
    

    
    \hypertarget{pruxe1ctica-1}{%
\section{Práctica 1}\label{pruxe1ctica-1}}

    \hypertarget{problema-1}{%
\subsection{Problema 1:}\label{problema-1}}

Considera los datos de la liga NBA de baloncesto profesional de la
temporada 2014-2015 que figuran en el archivo players\_stats.csv

    \begin{Verbatim}[commandchars=\\\{\}]
{\color{incolor}In [{\color{incolor}7}]:} \PY{c+c1}{\PYZsh{}\PYZsh{}Leemos el dataset}
        path \PY{o}{=} \PY{k+kp}{file.path}\PY{p}{(}\PY{l+s}{\PYZdq{}}\PY{l+s}{C:\PYZdq{}}\PY{p}{,} \PY{l+s}{\PYZdq{}}\PY{l+s}{Users\PYZdq{}}\PY{p}{,} \PY{l+s}{\PYZdq{}}\PY{l+s}{sergr\PYZdq{}}\PY{p}{,} \PY{l+s}{\PYZdq{}}\PY{l+s}{OneDrive\PYZdq{}}\PY{p}{,} \PY{l+s}{\PYZdq{}}\PY{l+s}{Universidad\PYZdq{}}\PY{p}{,} \PY{l+s}{\PYZdq{}}\PY{l+s}{MasterDataScience\PYZdq{}}\PY{p}{,} \PY{l+s}{\PYZdq{}}\PY{l+s}{NotebooksLocal\PYZdq{}}\PY{p}{,} \PY{l+s}{\PYZdq{}}\PY{l+s}{players\PYZus{}stats.csv\PYZdq{}}\PY{p}{)}
        nba \PY{o}{=} read.csv\PY{p}{(}file \PY{o}{=} path\PY{p}{,} header \PY{o}{=} \PY{k+kc}{TRUE}\PY{p}{,} sep \PY{o}{=} \PY{l+s}{\PYZdq{}}\PY{l+s}{,\PYZdq{}}\PY{p}{)}
        
        \PY{c+c1}{\PYZsh{} a) Hacemos el histograma}
        hist\PY{p}{(}nba\PY{o}{\PYZdl{}}Height\PY{p}{,} xlab \PY{o}{=} \PY{l+s}{\PYZdq{}}\PY{l+s}{Altura (cm)\PYZdq{}}\PY{p}{,} ylab\PY{o}{=} \PY{l+s}{\PYZdq{}}\PY{l+s}{Frecuencia\PYZdq{}}\PY{p}{,} main \PY{o}{=} \PY{l+s}{\PYZdq{}}\PY{l+s}{Altura de jugadores de la NBA\PYZdq{}}\PY{p}{)}
\end{Verbatim}


    \begin{center}
    \adjustimage{max size={0.9\linewidth}{0.9\paperheight}}{output_2_0.png}
    \end{center}
    { \hspace*{\fill} \\}
    
    \begin{Verbatim}[commandchars=\\\{\}]
{\color{incolor}In [{\color{incolor}6}]:} \PY{c+c1}{\PYZsh{} b) Calculamos la media}
        meanNba \PY{o}{=} \PY{k+kp}{mean}\PY{p}{(}nba\PY{o}{\PYZdl{}}Height\PY{p}{,} na.rm \PY{o}{=} \PY{k+kc}{TRUE}\PY{p}{)}
\end{Verbatim}


    \begin{Verbatim}[commandchars=\\\{\}]
{\color{incolor}In [{\color{incolor}8}]:} \PY{c+c1}{\PYZsh{} c) Calculamos la desviación estandar }
        stdNba \PY{o}{=} sd\PY{p}{(}nba\PY{o}{\PYZdl{}}Height\PY{p}{,} na.rm \PY{o}{=} \PY{k+kc}{TRUE}\PY{p}{)}
        
        \PY{c+c1}{\PYZsh{}Ploteamos}
        hist\PY{p}{(}nba\PY{o}{\PYZdl{}}Height\PY{p}{,} xlab \PY{o}{=} \PY{l+s}{\PYZdq{}}\PY{l+s}{Altura (cm)\PYZdq{}}\PY{p}{,} ylab\PY{o}{=} \PY{l+s}{\PYZdq{}}\PY{l+s}{Frecuencia\PYZdq{}}\PY{p}{,} main \PY{o}{=} \PY{l+s}{\PYZdq{}}\PY{l+s}{Altura de jugadores de la NBA\PYZdq{}}\PY{p}{)}
        abline\PY{p}{(}v \PY{o}{=} meanNba\PY{p}{,} col \PY{o}{=} \PY{l+s}{\PYZdq{}}\PY{l+s}{red\PYZdq{}}\PY{p}{,} lwd \PY{o}{=} \PY{l+m}{3}\PY{p}{)}
        abline\PY{p}{(}v \PY{o}{=} meanNba \PY{o}{\PYZhy{}} stdNba\PY{p}{,}
               col \PY{o}{=} \PY{l+s}{\PYZdq{}}\PY{l+s}{blue\PYZdq{}}\PY{p}{,} lwd \PY{o}{=} \PY{l+m}{3}\PY{p}{,} lty \PY{o}{=} \PY{l+m}{2}\PY{p}{)}
        abline\PY{p}{(}v \PY{o}{=} meanNba \PY{o}{+} stdNba\PY{p}{,}
               col \PY{o}{=} \PY{l+s}{\PYZdq{}}\PY{l+s}{blue\PYZdq{}}\PY{p}{,} lwd \PY{o}{=} \PY{l+m}{3}\PY{p}{,} lty \PY{o}{=} \PY{l+m}{2}\PY{p}{)}
        
        \PY{c+c1}{\PYZsh{}Sacamos la leyenda}
        legend\PY{p}{(}\PY{l+s}{\PYZdq{}}\PY{l+s}{topright\PYZdq{}}\PY{p}{,} \PY{k+kt}{c}\PY{p}{(}\PY{l+s}{\PYZdq{}}\PY{l+s}{Mean\PYZdq{}}\PY{p}{,} \PY{l+s}{\PYZdq{}}\PY{l+s}{Std Dev\PYZdq{}}\PY{p}{)}\PY{p}{,} col\PY{o}{=}\PY{k+kt}{c}\PY{p}{(}\PY{l+s}{\PYZdq{}}\PY{l+s}{red\PYZdq{}}\PY{p}{,} \PY{l+s}{\PYZdq{}}\PY{l+s}{blue\PYZdq{}}\PY{p}{)}\PY{p}{,} lwd\PY{o}{=}\PY{l+m}{5}\PY{p}{)}
\end{Verbatim}


    \begin{center}
    \adjustimage{max size={0.9\linewidth}{0.9\paperheight}}{output_4_0.png}
    \end{center}
    { \hspace*{\fill} \\}
    
    \hypertarget{d}{%
\subsubsection{d)}\label{d}}

En la gráfica no se puede apreciar ninguna distribución conocida
(gaussiana, poissoiniana, etc\ldots{}) a simple vista. La media se
encuentra cerca de los dos metros de altura, y la mayoría de jugadores
entran dentro del rango establecido por la desviación estándar
\(\sigma\).

    \hypertarget{problema-2}{%
\subsection{Problema 2:}\label{problema-2}}

Datos de extensión de hielo ártico.

    \begin{Verbatim}[commandchars=\\\{\}]
{\color{incolor}In [{\color{incolor}10}]:} \PY{c+c1}{\PYZsh{}\PYZsh{} a) Eliminamos la segunda línea}
         \PY{k+kp}{setwd}\PY{p}{(}\PY{l+s}{\PYZdq{}}\PY{l+s}{C:/Users/sergr/OneDrive/Universidad/MasterDataScience/NotebooksLocal\PYZdq{}}\PY{p}{)}
         all\PYZus{}content \PY{o}{\PYZlt{}\PYZhy{}} \PY{k+kp}{readLines}\PY{p}{(}\PY{l+s}{\PYZdq{}}\PY{l+s}{N\PYZus{}seaice\PYZus{}extent\PYZus{}daily\PYZus{}v3.0.csv\PYZdq{}}\PY{p}{)}
         
         skip\PYZus{}second \PY{o}{\PYZlt{}\PYZhy{}} all\PYZus{}content\PY{p}{[}\PY{l+m}{\PYZhy{}2}\PY{p}{]}
         
         seaice \PY{o}{\PYZlt{}\PYZhy{}} read.csv\PY{p}{(}\PY{k+kp}{textConnection}\PY{p}{(}skip\PYZus{}second\PY{p}{)}\PY{p}{)}
\end{Verbatim}


    \begin{Verbatim}[commandchars=\\\{\}]
{\color{incolor}In [{\color{incolor}22}]:} \PY{c+c1}{\PYZsh{}b) Buscamos los meses cuando la extensión es máxima y mínima para cada año}
         
         l \PY{o}{=} \PY{k+kp}{max}\PY{p}{(}seaice\PY{o}{\PYZdl{}}Year\PY{p}{)} \PY{o}{\PYZhy{}} \PY{k+kp}{min}\PY{p}{(}seaice\PY{o}{\PYZdl{}}Year\PY{p}{)} \PY{o}{\PYZhy{}} \PY{l+m}{1} \PY{c+c1}{\PYZsh{}Longitud del vector de resultados }
         monthMin \PY{o}{=} \PY{k+kt}{numeric}\PY{p}{(}length \PY{o}{=} l\PY{p}{)}
         monthMax \PY{o}{=} \PY{k+kt}{numeric}\PY{p}{(}length \PY{o}{=} l\PY{p}{)}
         
         years \PY{o}{=} \PY{k+kp}{seq}\PY{p}{(}\PY{k+kp}{min}\PY{p}{(}seaice\PY{o}{\PYZdl{}}Year\PY{p}{)} \PY{o}{+} \PY{l+m}{1}\PY{p}{,} \PY{k+kp}{max}\PY{p}{(}seaice\PY{o}{\PYZdl{}}Year\PY{p}{)} \PY{o}{\PYZhy{}} \PY{l+m}{1}\PY{p}{)} \PY{c+c1}{\PYZsh{}Ignoramos el primer año (1978)}
                                                                 \PY{c+c1}{\PYZsh{}y el último (2011)}
                                                                 \PY{c+c1}{\PYZsh{}porque tienen datos incompletos}
         
         
         \PY{k+kr}{for}\PY{p}{(}i \PY{k+kr}{in} years\PY{p}{)}\PY{p}{\PYZob{}}
           fixedYear \PY{o}{=} \PY{k+kp}{subset}\PY{p}{(}seaice\PY{p}{,} Year \PY{o}{==} i\PY{p}{)}
           monthMin\PY{p}{[}i \PY{o}{\PYZhy{}} \PY{k+kp}{min}\PY{p}{(}seaice\PY{o}{\PYZdl{}}Year\PY{p}{)}\PY{p}{]} \PY{o}{=} fixedYear\PY{o}{\PYZdl{}}Month\PY{p}{[}\PY{k+kp}{which.min}\PY{p}{(}fixedYear\PY{o}{\PYZdl{}}Extent\PY{p}{)}\PY{p}{]}
           monthMax\PY{p}{[}i \PY{o}{\PYZhy{}} \PY{k+kp}{min}\PY{p}{(}seaice\PY{o}{\PYZdl{}}Year\PY{p}{)}\PY{p}{]} \PY{o}{=} fixedYear\PY{o}{\PYZdl{}}Month\PY{p}{[}\PY{k+kp}{which.max}\PY{p}{(}fixedYear\PY{o}{\PYZdl{}}Extent\PY{p}{)}\PY{p}{]}
         \PY{p}{\PYZcb{}}
         
         plot\PY{p}{(}x \PY{o}{=} years\PY{p}{,} y \PY{o}{=} monthMax\PY{p}{,} ylim \PY{o}{=} \PY{k+kt}{c}\PY{p}{(}\PY{l+m}{0}\PY{p}{,}\PY{l+m}{3.5}\PY{p}{)}\PY{p}{,} cex \PY{o}{=} \PY{l+m}{1}\PY{p}{,} pch \PY{o}{=} \PY{l+m}{16}\PY{p}{,}
              xlab \PY{o}{=} \PY{l+s}{\PYZdq{}}\PY{l+s}{Año\PYZdq{}}\PY{p}{,} ylab\PY{o}{=} \PY{l+s}{\PYZdq{}}\PY{l+s}{Mes máximo\PYZdq{}}\PY{p}{,} main \PY{o}{=} \PY{l+s}{\PYZdq{}}\PY{l+s}{Mes en el que la extensión ha sido máxima\PYZdq{}}\PY{p}{)}
         plot\PY{p}{(}x \PY{o}{=} years\PY{p}{,} y \PY{o}{=} monthMin\PY{p}{,} ylim \PY{o}{=} \PY{k+kt}{c}\PY{p}{(}\PY{l+m}{0}\PY{p}{,}\PY{l+m}{10}\PY{p}{)}\PY{p}{,} cex \PY{o}{=} \PY{l+m}{1}\PY{p}{,} pch \PY{o}{=} \PY{l+m}{16}\PY{p}{,}
              xlab \PY{o}{=} \PY{l+s}{\PYZdq{}}\PY{l+s}{Año\PYZdq{}}\PY{p}{,} ylab\PY{o}{=} \PY{l+s}{\PYZdq{}}\PY{l+s}{Mes mínimo\PYZdq{}}\PY{p}{,} main \PY{o}{=} \PY{l+s}{\PYZdq{}}\PY{l+s}{Mes en el que la extensión ha sido mínima\PYZdq{}}\PY{p}{)}
\end{Verbatim}


    \begin{center}
    \adjustimage{max size={0.9\linewidth}{0.9\paperheight}}{output_8_0.png}
    \end{center}
    { \hspace*{\fill} \\}
    
    \begin{center}
    \adjustimage{max size={0.9\linewidth}{0.9\paperheight}}{output_8_1.png}
    \end{center}
    { \hspace*{\fill} \\}
    
    Como vemos, la extensión máxima se suele producir en Febrero o Marzo,
mientras que la extensión mínima siempre se produce en Septiembre.

    \begin{Verbatim}[commandchars=\\\{\}]
{\color{incolor}In [{\color{incolor}25}]:} \PY{c+c1}{\PYZsh{} c) Buscamos los medianos de cada año}
         months \PY{o}{=} \PY{k+kp}{seq}\PY{p}{(}\PY{l+m}{1}\PY{p}{,}\PY{l+m}{12}\PY{p}{)}
         median \PY{o}{=} \PY{k+kt}{numeric}\PY{p}{(}length \PY{o}{=} \PY{l+m}{12}\PY{p}{)}
         
         \PY{k+kr}{for} \PY{p}{(}i \PY{k+kr}{in} \PY{k+kp}{months}\PY{p}{)}\PY{p}{\PYZob{}}
           fixedMonth \PY{o}{=} subset \PY{p}{(}seaice\PY{p}{,} Month \PY{o}{==} i\PY{p}{)}
           median\PY{p}{[}i\PY{p}{]} \PY{o}{=} median\PY{p}{(}fixedMonth\PY{o}{\PYZdl{}}Extent\PY{p}{)}
         \PY{p}{\PYZcb{}}
         
         plot\PY{p}{(}\PY{k+kp}{months}\PY{p}{,} median\PY{p}{,} ylim \PY{o}{=} \PY{k+kt}{c}\PY{p}{(}\PY{k+kp}{min}\PY{p}{(}p5\PY{p}{)}\PY{p}{,}\PY{k+kp}{max}\PY{p}{(}p95\PY{p}{)}\PY{p}{)}\PY{p}{,} cex \PY{o}{=} \PY{l+m}{1}\PY{p}{,} pch \PY{o}{=} \PY{l+m}{16}\PY{p}{,}
              xlab \PY{o}{=} \PY{l+s}{\PYZdq{}}\PY{l+s}{Mes\PYZdq{}}\PY{p}{,} ylab\PY{o}{=} \PY{l+s}{\PYZdq{}}\PY{l+s}{Mediana\PYZdq{}}\PY{p}{,} main \PY{o}{=} \PY{l+s}{\PYZdq{}}\PY{l+s}{Valor de la mediana en un ciclo anual\PYZdq{}}\PY{p}{)}
\end{Verbatim}


    \begin{center}
    \adjustimage{max size={0.9\linewidth}{0.9\paperheight}}{output_10_0.png}
    \end{center}
    { \hspace*{\fill} \\}
    
    \begin{Verbatim}[commandchars=\\\{\}]
{\color{incolor}In [{\color{incolor}28}]:} \PY{c+c1}{\PYZsh{} d) Añadimos región sombreada para los percentiles}
         p5 \PY{o}{=} \PY{k+kt}{numeric}\PY{p}{(}length \PY{o}{=} \PY{l+m}{12}\PY{p}{)}
         p95 \PY{o}{=} \PY{k+kt}{numeric}\PY{p}{(}length \PY{o}{=} \PY{l+m}{12}\PY{p}{)}
         
         \PY{k+kr}{for} \PY{p}{(}i \PY{k+kr}{in} \PY{k+kp}{months}\PY{p}{)}\PY{p}{\PYZob{}}
           fixedMonth \PY{o}{=} subset \PY{p}{(}seaice\PY{p}{,} Month \PY{o}{==} i\PY{p}{)}
           p5\PY{p}{[}i\PY{p}{]} \PY{o}{=} quantile\PY{p}{(}fixedMonth\PY{o}{\PYZdl{}}Extent\PY{p}{,} \PY{l+m}{0.05}\PY{p}{)}
           p95\PY{p}{[}i\PY{p}{]} \PY{o}{=} quantile\PY{p}{(}fixedMonth\PY{o}{\PYZdl{}}Extent\PY{p}{,} \PY{l+m}{0.95}\PY{p}{)}
         \PY{p}{\PYZcb{}}
         
         plot\PY{p}{(}\PY{k+kp}{months}\PY{p}{,} median\PY{p}{,} ylim \PY{o}{=} \PY{k+kt}{c}\PY{p}{(}\PY{k+kp}{min}\PY{p}{(}p5\PY{p}{)}\PY{p}{,}\PY{k+kp}{max}\PY{p}{(}p95\PY{p}{)}\PY{p}{)}\PY{p}{,} cex \PY{o}{=} \PY{l+m}{1}\PY{p}{,} pch \PY{o}{=} \PY{l+m}{16}\PY{p}{,}
              xlab \PY{o}{=} \PY{l+s}{\PYZdq{}}\PY{l+s}{Mes\PYZdq{}}\PY{p}{,} ylab\PY{o}{=} \PY{l+s}{\PYZdq{}}\PY{l+s}{Mediana\PYZdq{}}\PY{p}{,} main \PY{o}{=} \PY{l+s}{\PYZdq{}}\PY{l+s}{Valor de la mediana en un ciclo anual\PYZdq{}}\PY{p}{)}
         lines\PY{p}{(}\PY{k+kp}{months}\PY{p}{,} p5\PY{p}{,} col \PY{o}{=} \PY{l+s}{\PYZdq{}}\PY{l+s}{blue\PYZdq{}}\PY{p}{)}
         lines\PY{p}{(}\PY{k+kp}{months}\PY{p}{,} p95\PY{p}{,} col \PY{o}{=} \PY{l+s}{\PYZdq{}}\PY{l+s}{blue\PYZdq{}}\PY{p}{)}
         polygon\PY{p}{(}\PY{k+kt}{c}\PY{p}{(}\PY{k+kp}{months}\PY{p}{,} \PY{k+kp}{rev}\PY{p}{(}\PY{k+kp}{months}\PY{p}{)}\PY{p}{)}\PY{p}{,} \PY{k+kt}{c}\PY{p}{(}p95\PY{p}{,}\PY{k+kp}{rev}\PY{p}{(}p5\PY{p}{)}\PY{p}{)}\PY{p}{,} density \PY{o}{=} \PY{l+m}{15}\PY{p}{,} col \PY{o}{=} \PY{l+s}{\PYZdq{}}\PY{l+s}{blue\PYZdq{}}\PY{p}{,} border \PY{o}{=} \PY{k+kc}{NA}\PY{p}{)}
         
         
         \PY{c+c1}{\PYZsh{}Para polygon, primero se definen todos los puntos de \PYZdq{}arriba\PYZdq{} y al llegar al final}
         \PY{c+c1}{\PYZsh{}tiene que volver por el eje x para definir los puntos de \PYZdq{}abajo\PYZdq{}, por eso }
         \PY{c+c1}{\PYZsh{}tenemos que poner el reverse()}
\end{Verbatim}


    \begin{center}
    \adjustimage{max size={0.9\linewidth}{0.9\paperheight}}{output_11_0.png}
    \end{center}
    { \hspace*{\fill} \\}
    
    \begin{Verbatim}[commandchars=\\\{\}]
{\color{incolor}In [{\color{incolor}34}]:} \PY{c+c1}{\PYZsh{} e) Añadimos datos para 2012 y 2021}
         
         y12 \PY{o}{=} \PY{k+kp}{subset}\PY{p}{(}seaice\PY{p}{,} Year \PY{o}{==} \PY{l+m}{2012}\PY{p}{)}
         y21 \PY{o}{=} \PY{k+kp}{subset}\PY{p}{(}seaice\PY{p}{,} Year \PY{o}{==} \PY{l+m}{2021}\PY{p}{)}
         m12 \PY{o}{=} \PY{k+kt}{c}\PY{p}{(}\PY{p}{)}
         m21 \PY{o}{=} \PY{k+kt}{c}\PY{p}{(}\PY{p}{)}
         \PY{k+kr}{for}\PY{p}{(}i \PY{k+kr}{in} \PY{k+kp}{months}\PY{p}{)}\PY{p}{\PYZob{}}
           fixedMonth12 \PY{o}{=} \PY{k+kp}{subset}\PY{p}{(}y12\PY{p}{,} Month \PY{o}{==} i\PY{p}{)}
           m12\PY{p}{[}i\PY{p}{]} \PY{o}{=} median\PY{p}{(}fixedMonth12\PY{o}{\PYZdl{}}Extent\PY{p}{)}
           fixedMonth21 \PY{o}{=} \PY{k+kp}{subset}\PY{p}{(}y21\PY{p}{,} Month \PY{o}{==} i\PY{p}{)}
           m21\PY{p}{[}i\PY{p}{]} \PY{o}{=} median\PY{p}{(}fixedMonth21\PY{o}{\PYZdl{}}Extent\PY{p}{)}
         \PY{p}{\PYZcb{}}
         
         \PY{c+c1}{\PYZsh{} Repetimos el plot anterior}
         plot\PY{p}{(}\PY{k+kp}{months}\PY{p}{,} median\PY{p}{,} ylim \PY{o}{=} \PY{k+kt}{c}\PY{p}{(}\PY{k+kp}{min}\PY{p}{(}p5\PY{p}{)}\PY{p}{,}\PY{k+kp}{max}\PY{p}{(}p95\PY{p}{)}\PY{p}{)}\PY{p}{,} cex \PY{o}{=} \PY{l+m}{1}\PY{p}{,} pch \PY{o}{=} \PY{l+m}{16}\PY{p}{,}
              xlab \PY{o}{=} \PY{l+s}{\PYZdq{}}\PY{l+s}{Mes\PYZdq{}}\PY{p}{,} ylab\PY{o}{=} \PY{l+s}{\PYZdq{}}\PY{l+s}{Mediana\PYZdq{}}\PY{p}{,} main \PY{o}{=} \PY{l+s}{\PYZdq{}}\PY{l+s}{Valor de la mediana en un ciclo anual\PYZdq{}}\PY{p}{)}
         lines\PY{p}{(}\PY{k+kp}{months}\PY{p}{,} p5\PY{p}{,} col \PY{o}{=} \PY{l+s}{\PYZdq{}}\PY{l+s}{blue\PYZdq{}}\PY{p}{)}
         lines\PY{p}{(}\PY{k+kp}{months}\PY{p}{,} p95\PY{p}{,} col \PY{o}{=} \PY{l+s}{\PYZdq{}}\PY{l+s}{blue\PYZdq{}}\PY{p}{)}
         polygon\PY{p}{(}\PY{k+kt}{c}\PY{p}{(}\PY{k+kp}{months}\PY{p}{,} \PY{k+kp}{rev}\PY{p}{(}\PY{k+kp}{months}\PY{p}{)}\PY{p}{)}\PY{p}{,} \PY{k+kt}{c}\PY{p}{(}p95\PY{p}{,}\PY{k+kp}{rev}\PY{p}{(}p5\PY{p}{)}\PY{p}{)}\PY{p}{,} density \PY{o}{=} \PY{l+m}{15}\PY{p}{,} col \PY{o}{=} \PY{l+s}{\PYZdq{}}\PY{l+s}{blue\PYZdq{}}\PY{p}{,} border \PY{o}{=} \PY{k+kc}{NA}\PY{p}{)}
         
         \PY{c+c1}{\PYZsh{} Ploteamos datos para 2012 y 2021}
         lines\PY{p}{(}\PY{k+kp}{months}\PY{p}{,} m12\PY{p}{,} col \PY{o}{=} \PY{l+s}{\PYZdq{}}\PY{l+s}{red\PYZdq{}}\PY{p}{,} lwd \PY{o}{=} \PY{l+m}{2}\PY{p}{)}
         lines\PY{p}{(}\PY{k+kp}{months}\PY{p}{,} m21\PY{p}{,} col \PY{o}{=} \PY{l+s}{\PYZdq{}}\PY{l+s}{green\PYZdq{}}\PY{p}{,} lwd \PY{o}{=} \PY{l+m}{2}\PY{p}{)}
         legend\PY{p}{(}\PY{l+s}{\PYZdq{}}\PY{l+s}{bottomleft\PYZdq{}}\PY{p}{,} \PY{k+kt}{c}\PY{p}{(}\PY{l+s}{\PYZdq{}}\PY{l+s}{2012\PYZdq{}}\PY{p}{,} \PY{l+s}{\PYZdq{}}\PY{l+s}{2021\PYZdq{}}\PY{p}{)}\PY{p}{,} col \PY{o}{=} \PY{k+kt}{c}\PY{p}{(}\PY{l+s}{\PYZdq{}}\PY{l+s}{red\PYZdq{}}\PY{p}{,} \PY{l+s}{\PYZdq{}}\PY{l+s}{green\PYZdq{}}\PY{p}{)}\PY{p}{,} lwd \PY{o}{=} \PY{l+m}{4}\PY{p}{)}
\end{Verbatim}


    \begin{center}
    \adjustimage{max size={0.9\linewidth}{0.9\paperheight}}{output_12_0.png}
    \end{center}
    { \hspace*{\fill} \\}
    
    En la gráfica podemos observar como los datos de
\(\color{green}{\text{2021}}\) caen íntegramente dentro de el rango
establecido por los percentiles del 5\% y 95\%. Sin embargo, los datos
de \(\color{red}{\text{2012}}\) abandonan el rango en los meses de
Agosto y Septiembre. Esto nos indica una anomalía: al ser menor la
extensión de hielo en Septiembre en comparación con la dinámica anual,
\(\color{red}{\text{2012}}\) podría identificarse como ``inusualmente
cálido'' durante esos meses.


    % Add a bibliography block to the postdoc
    
    
    
    \end{document}
